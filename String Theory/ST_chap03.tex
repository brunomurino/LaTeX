\documentclass[oneside, 12pt]{book}

\usepackage{mypreamble}
\usepackage[backend=biber,style=nature]{biblatex}
%\cite{barton} = [1] and \footfullcite[p. 324]{barton} = ^1 + footnote, p. 324 (e.g.)

\usepackage{mycommands}
\usepackage{mytheme1}

\addbibresource{ref_ST.bib}

\begin{document}



\chapter{Fields}

\section{Scalar fields}
Let \(\phi : \mb{R}^D \mapsto \mb{R}\) be a scalar field where \(D\) is the spacetime dimension. Its Lagrangian \(\mc{L}\) is declared to be
\beq \mc{L} =  -\hlf (\del \phi)^2 - \hlf m^2 \phi^2 = -\hlf(\del_0\phi)^2 - \hlf(\gradi{\phi})^2 - \hlf m^2\phi^2\eeq
noting that it is Lorentz invariant. We associate the kinetic energy with the \( (\del_0\phi)^2\) term, this means
\beq T = \hlf(\del_0 \phi)^2\eeq
therefore the conjugate momentum \(\Pi\) is
\beq \Pi = \pdv{\mc{L}}{(\del_0\phi)} = \del_0\phi \eeq
hence the Hamiltonian \( \mc{H}\) is
\beq \mc{H} = \hlf \Pi^2 + \hlf (\gradi{\phi})^2 + \hlf m^2\phi^2 \eeq
The Lagrangian gives rise to the Klein-Gordon equation
\beq (\del^2 - m^2)\phi =0\eeq
or more explicitly
\beq -\pdv[2]{\phi}{t} + \laplacian \phi -m^2 \phi = 0\eeq

\section{Quantum fields}

One-particle states of a scalar field are
\beq[eq:opsofascalarfield] a^{\dag}_{p^{+},p_T}\ket{\Omega} \eeq

\section{Maxwell's fields}

\section{Gravitational fields}



\end{document}