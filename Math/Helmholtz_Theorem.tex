\documentclass[oneside, 12pt]{book}

\usepackage{mypreamble}
\usepackage[backend=biber,style=nature]{biblatex}
%\cite{barton} = [1] and \footfullcite[p. 324]{barton} = ^1 + footnote, p. 324 (e.g.)

\usepackage{mycommands}
\usepackage{mytheme1}

\addbibresource{ref_FD.bib}

%--------------------------------------------------------------



%--------------------------------------------------------------

\begin{document}

\mainmatter \pagestyle{mypage2} \normalfont

\chapter{The Helmholtz Theorem} 

Let \(\vb{F}(\vb{x})\) be an unknown vector field with given 
\beq[eq:helmdivF] D = \div{\vb{F}}\eeq 
and
\beq[eq:helmcurlF] \vb{C} = \curl{\vb{F}}\eeq
Using 
\beq[eq:deltapick] \int \dd{x}f(x)\delta(x - x') = f(x')\eeq
followed by
\beq[eq:divdelta] \div{\left(\frac{\vb{x}-\vb{x'}}{\abs{\vb{x}-\vb{x'}}^3}\right)} = 4\pi \delta(\vb{x}-\vb{x'})\eeq
and
\beq[eq:grad1r] \gradi{\left( \frac{1}{r}\right)} = -\frac{\vb{r}}{r^3} = -\frac{\hat{r}}{r^2}\eeq
we have
\beq[eq:helm1] \vb{F}(\vb{x}) = -\frac{\laplacian_{x}}{4\pi} \int \dd[3]{x'} \left( \frac{\vb{F}(\vb{x'})}{\abs{\vb{x}-\vb{x'}}}\right) \eeq
now, using
\beq[eq:curlcurl] \curl{\left(\curl{\vb{V}}\right)} = \gradi{\left(\div{\vb{V}}\right)} - \laplacian{\vb{V}}\eeq
to replace the laplacian, we have
\beq[eq:helm2] \vb{F}(\vb{x}) = -\frac{1}{4\pi}\gradi[x]{A} + \frac{1}{4\pi} \curl[x]{B}\eeq
where
\beq[eq:helmA] A = \gradi[x]{\left( \int \dd[3]{x'} \frac{\vb{F}(\vb{x'})}{\abs{\vb{x}-\vb{x'}}}\right)} \eeq 
and
\beq[eq:helmB] \vb{B} = \curl[x]{\left( \int \dd[3]{x'}\frac{\vb{F}(\vb{x'})}{\abs{\vb{x} - \vb{x'}}}\right)}\eeq
Putting the gradient inside the integral in \eqref{eq:helmA}, using
\beq[eq:gradxx'] \gradi[x]{\left( \frac{1}{\abs{\vb{x}-\vb{x'}}}\right)} = -\gradi[x']{\left( \frac{1}{\abs{\vb{x}-\vb{x'}}}\right)}\eeq
and
\beq[eq:divaV] \div{\left( a\vb{V}\right)} = \left(\gradi{a} \right)\cdot\vb{V} + a\left(\div{\vb{V}}\right)\eeq
we find that
\beq[eq:helmA2] A = \int \dd[3]{x'}\frac{\div[x']{\vb{F}}\left(\vb{x'}\right)}{\abs{\vb{x}-\vb{x'}}} - \int \dd[3]{x'} \div[x']{\left( \frac{\vb{F}\left( \vb{x'}\right)}{\abs{\vb{x}-\vb{x'}}}\right)}\eeq
Now, substituting \eqref{eq:helmdivF} and using Stokes' theorem on \eqref{eq:helmA2}, we find that
\beq[eq:helmAfinal] A = \und{\mc{V}}{\int} \dd[3]{x'} \frac{D(x')}{\abs{\vb{x} - \vb{x'}}} - \und{\del \mc{V}}{\oint} \dd{\vb{S}_{x'}}\cdot \left( \frac{\vb{F} \left(\vb{x'} \right)}{\abs{\vb{x} - \vb{x'}}}\right)\eeq
Similarly, putting the curl inside the integral in \eqref{eq:helmB} with \eqref{eq:gradxx'} and then
\beq[eq:curlaV] \curl{a\vb{V}} = \left( \gradi{a}\right)\cross \vb{V} + a \div{\vb{V}}\eeq
we have
\beq[eq:helmB2] \vb{B} =  \int \dd[3]{x'} \frac{\curl[x']{\vb{F}(\vb{x'})}}{\abs{\vb{x}-\vb{x'}}} - \int \dd[3]{x'} \curl[x']{\left( \frac{\vb{F}(\vb{x'})}{\abs{\vb{x}-\vb{x'}}}\right)}\eeq
Now, substituting \eqref{eq:helmcurlF} and using Stokes' theorem on \eqref{eq:helmB2}, we find that
\beq[eq:helmBfinal] \vb{B} = \und{\mc{V}}{\int} \dd[3]{x'} \frac{\vb{C}(x')}{\abs{\vb{x}-\vb{x'}}} - \und{\del\mc{V}}{\oint} \dd{\vb{S}_{x'}}\cross \left(\frac{\vb{F}(\vb{x'})}{\abs{\vb{x}-\vb{x'}}} \right) \eeq \par 
What all this means is that the curl and the divergence of an unknown vector field are not enough to fully determine the field, we must also know the field on the boundary of the region of integration, which is usually said as: we need to impose boundary conditions.\par


\end{document}