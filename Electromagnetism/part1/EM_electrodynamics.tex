\documentclass[oneside, 12pt, notitlepage]{book}

\usepackage{../../_mypackages/mypreamble}
\usepackage{../../_mypackages/mycommands}
\usepackage{../../_mythemes/notestheme}

\addbibresource{EM_ref.bib}

%----------------------------------END PREAMBLE---------------------------------

\begin{document}

\frontmatter
% \pagestyle{empty}
% \notestp{Electrodynamics}
% \begin{notesabstract}[Abstract]
% Abstract text
% \end{notesabstract}
% \centeredtoc{Content}
\mainmatter
\pagestyle{mynotespage} %\normalfont

\chapter{Electrodynamics}

We have Faraday's Law in differential form
\beq[] \curl{\vb{E}} = - \pdv{\vb{B}}{t} \eeq
which is always valid, and we have it in integral form
\beq[eq:intfaraday] \varepsilon \at{C} = \uoint[\del{S}] \dd{\vb{l}} \cdot \vb{E} = - \dv{\Phi_B}{t} \at{S} \eeq
which is valid in low speeds and \(c = \del{S}\). The minus sign in \eqref{eq:intfaraday} is called Lenz Law and \( \Phi_B \at{S}\) is the magnetic flux trough the surface \(S\) enclosed by the circuit \(C\) and is given by
\beq[]  \Phi_B \at{S} = \uint[S]\vb{B}\cdot \dd{\vb{S}}\eeq

\end{document}
