\documentclass[oneside, 12pt, notitlepage]{book}

\usepackage{../_mypackages/mypreamble}
\usepackage{../_mypackages/mycommands}
\usepackage{../_mythemes/notestheme}

\addbibresource{QM_ref.bib}

%----------------------------------END PREAMBLE---------------------------------

\begin{document}
\pagestyle{mynotespage}

\chapter{The WKB approximation}

\section{The classical region}

Let \(p(x)\in \R\)
\eq[eq:tise]{p(x) \equiv \sqrt{2m\left[E-V(x)\right]}}
be the value of the \emph{classical} momentum of a particle with total energy \(E\) and potential energy \(V(x)\). Then we can write Schrödinger equation as
\eq{\dv[2]{\psi}{x} = -\frac{p^2}{\hbar^2}\psi}
\par

The wave-function is a complex function, so we can write it as
\eq[eq:complexpsi]{\psi(x) = A(x) \exp{i\phi(x)}}
where \(A(x)\) and \(\phi(x)\) are real-valued functions.\par

Plugging \eqref{eq:complexpsi} into \eqref{eq:tise} we find two equations
\eq{A'' = A\left[(\phi')^2-\frac{p^2}{\hbar^2} \right]}
and
\eq{\lr{A^2\phi'}'=0}
While we can easily solve the second equation, namely
\eq{A =\frac{C}{\sqrt{\phi'}} \qq{with} C\in \R}
the other equation is not trivial. In fact, it cannot be solved without approximations.\par

Now, the so called WKB approximation consists in assuming that \(V(x)\) varies slowly, implying that the amplitude \(A\) varies slowly, such that
\eq{\lr{\phi'}^2 - \frac{p^2}{\hbar^2} - \frac{A''}{A} = 0 \approx \lr{\phi'}^2 - \frac{p^2}{\hbar^2} = 0 \longrightarrow \phi' =\pm \frac{p(x)}{\hbar}}
which leads to the solution
\eq{\phi(x) = \pm \frac{1}{\hbar}\int^x p(x)\dd{x}}\par

Of course, if \(V(x)\) doesn't vary slowly, the WKB approximation breaks down, and this surely happens at the \emph{turning points}! This situation will be addressed later.\par

It follows that the full wave-function in the WKB approximation is
\eq{\psi(x) \cong \frac{1}{\sqrt{p(x)}}\left[C_{+}\exp{+ i\phi(x) } + C_{-}\exp{- i\phi(x) } \right]  }
or, more conveniently,
\eq{\psi(x) \cong \frac{1}{\sqrt{p(x)}}\left[C_{1}\sin{\phi(x)} + C_{2}\cos{\phi(x)} \right]  }
with
\eq{\phi(x) = \frac{1}{\hbar} \int^x p(x)\dd{x}}\par

\begin{example}[Potential well with two vertical walls]

\end{example}


\section{Tunneling}

Let the potential be
\eq{V(x) = \left\{ \begin{aligned}
V_0(x) > 0 &\qq{for} 0 \leq x \leq a \\
0 &\qq{otherwise}
\end{aligned} \right.}
Then for \(x\in[0,a]\) let \(0 < E < V_0(x)\), then \(p(x)\) is imaginary and the wave-function becomes
\eq{\psi(x) \cong \frac{C}{\sqrt{\abs{p(x)}}} \lexp{\pm \frac{1}{\hbar}\int^x \abs{p(x')}\dd{x'}}}
For \(x<0\) we find
\eq{\psi(x) = A\exp{ikx} + B\exp{-ikx}}
and for \(x>a\) we find
\eq{\psi(x) = F \exp{ikx}}
For the non-classical region \(x\in [0,a]\), we use the WKB approximation to find
\eq{\psi(x) \cong \frac{C}{\sqrt{\abs{p(x)}}} \exp{\int^x \abs{p(x')}\dd{x'}/\hbar} + \frac{D}{\sqrt{\abs{p(x)}}} \exp{-\int^x \abs{p(x')}\dd{x'}/\hbar}   }\par

The transmission coefficient \(T\)
\eq{T = \frac{\abs{F}^2}{\abs{A}^2}}
is then given by
\eq{T \cong \exp{-2\gamma} \qq{with} \gamma \equiv \frac{1}{\hbar} \int_0^a \abs{p(x')}\dd{x'}}

\begin{example}[Gamow's theory of alpha decay]


\end{example}

\section{The connection formulas}

Now comes the time to study what happens at the turning points, where the classical region joins the non-classical region.\par


Let the energt of the particle be constant \(E\). Let \(V(x)\) be linearly increasing around \(x=0\), and specifically let \(E > V(x)\) for \(x<0\), \(E < V(x)\) for \(x>0\), and \(E = V(0)\). Note that \(p(0)\) would be \(0\) and \(\psi(x)\) would diverge.\par

What we'll do to fix this problem is to create a \emph{patching function} \(\psi_p\) such that \(\psi(x) = \psi_p(x)\) near \(x=0\). Near \(x=0\) we can write \(V(x) = E + V'(0)x\), such that
\eq{\left[ -\frac{\hbar^2}{2m}\dv[2]{x} + \lr{E+V'(0)x}\right]\psi_p(x) = E \psi_p(x)}
we can then write
\eq{\dv[2]{\psi_p}{x} = \alpha^3 x\ \psi_p \qq{where} \alpha^3 = \left[\frac{2m}{\hbar^2}V'(0)\right]}
and by defining \(z\equiv \alpha x\) we can write
\eq{\dv[2]{\psi_p}{z} = z \psi_p}
which is the so called \emph{Airy's equation}, from which the solution are the \emph{Airy functions}, meaning
\eq{\psi_p = a \Ai{\alpha x} + b\Bi{\alpha x}}\par
Lets state the asymptotic forms of the Airy functions, since we'll be using them later:
\eq{\left.
\begin{aligned}
	\Ai{z}&\sim \frac{1}{2z^{1/4}\sqrt{\pi}}\lexp{-\frac{2}{3}z^{3/2}}\\
	\Bi{z}&\sim \frac{1}{z^{1/4}\sqrt{\pi}}\lexp{\frac{2}{3}z^{1/2}}
\end{aligned}
\right\} z \gg 0}
\eq{\left.
\begin{aligned}
	\Ai{z}&\sim \frac{1}{\lr{-z}^{1/4}\sqrt{\pi}}\sin{\left[\frac{2}{3}\lr{-z}^{3/2} + \frac{\pi}{4}\right]}\\
	\Bi{z}&\sim \frac{1}{\lr{-z}^{1/4}\sqrt{\pi}}\cos{\left[\frac{2}{3}\lr{-z}^{3/2} + \frac{\pi}{4}\right]}
\end{aligned}
\right\} z \ll 0}\par

Now, we must assume that to the left of \(x=0\) it can happen that \(\alpha x \ll 0\) and \(V(x) \cong E + V'(0)x\) remains a good approximation to the potential. Analogously to the right of \(x=0\). This means that on such regions both the patching function \(\psi_p\) and \(\psi_{WKB}\) must match. Then \(p(x)\) can be written as
\eq{p(x) \cong \sqrt{2m\lr{E - \lr{E + V'(0)x}}} = \hbar \alpha^{3/2}\sqrt{-x}}
recalling that when \(x<0\) it follows that \(p(x)\) is real, whereas when \(x>0\) then \(p(x)\) is imaginary.\par

Since the region for which \(x>0\) is non-classical, the following is the solution via the WKB approximation:
\eq{\psi(x) \cong \frac{C}{\sqrt{\abs{p(x)}}} \exp{\int^x \abs{p(x')}\dd{x'}/\hbar} + \frac{D}{\sqrt{\abs{p(x)}}} \exp{-\int^x \abs{p(x')}\dd{x'}/\hbar}   }
and if we assume that \(E < V(x)\) for every \(x>0\), then we can drop the first term and get, using \(p(x) = \hbar \alpha^{3/2}\sqrt{-x}\)
\eq{\psi(x) \cong \frac{D}{\alpha^{3/4}x^{1/4}\sqrt{\hbar}}\lexp{-\frac{2}{3}\lr{\alpha x}^{3/2}}}
while the patching function becomes, recalling the \(z\gg0\) form
\eq{\psi_p \cong \frac{a}{2\lr{\alpha x}^{1/4}\sqrt{\pi}}\lexp{-\frac{2}{3}\lr{\alpha x}^{3/2}} + \frac{b}{\lr{\alpha x}^{1/4}\sqrt{\pi}}\lexp{\frac{2}{3}\lr{\alpha x}^{3/2}}}
and then, by comparison, it must follow that
\eq{a = D \sqrt{\frac{4\pi}{\alpha \hbar}} \qq{and} b = 0}\par


Now for the classical region \(x<0\) we recall that the solution via the WKB approximation is
\eq{\psi(x) \cong \frac{1}{\sqrt{p(x)}}\left[B\exp{+ i\frac{1}{\hbar} \int^x p(x)\dd{x} } + C\exp{- i\frac{1}{\hbar} \int^x p(x)\dd{x} } \right]  }
and plugging in \(p(x) = \hbar \alpha^{3/2}\sqrt{-x}\)
\eq{\psi(x) = \frac{1}{\lr{-x}^{1/4}\alpha^{3/4}\sqrt{\hbar}}\left[ B\ \lexp{i\frac{2}{3}\lr{-\alpha x}^{3/2}} + C\ \lexp{-i\frac{2}{3}\lr{-\alpha x}^{3/2}} \right] }
while the patching function becomes, recalling the \(z\ll 0\) form
\eq{\psi_p(x) \frac{a}{\lr{-\alpha x}^{1/4}\sqrt{\pi}}\frac{1}{2i} \left[ \exp{i\pi/4}\exp{i\frac{2}{3}\lr{-\alpha x}^{3/2}} - \exp{-i\pi/4}\exp{-i\frac{2}{3}\lr{-\alpha x}^{3/2}} \right]}
which, by comparison and recalling that we already know \(a\), implies
\eq{B = -iD\exp{i\pi/4} \qq{and} C = i D \exp{-i\pi/4}}
which are the so called \emph{connection formulas}.\par

The thing is that now we know that if \(B\) and \(C\) assume the values obtained above, in terms of \(D\), near the turning point \(x = x_2\), then the WKB solution will match the patching function, which we constructed to be a good approximation to the real wave-function and therefore works well! In terms of \(D\), the WKB solution becomes (on all space)
\eq{\psi(x) \cong \left\{
\begin{aligned}
	\frac{2D}{\sqrt{p(x)}} \sin \left[\frac{\pi}{4} + \frac{1}{\hbar}\int_x^{x_2}p(x')\dd{x'} \right] \qq{for} x < x_2 \\
	\frac{D}{\sqrt{\abs{p(x)}}} \lexp{ -\frac{1}{\hbar} \int_{x_2}^x \abs{p(x')}\dd{x'}} \qq{for} x > x_2
\end{aligned} \right. }

\chapter{Time-dependent perturbation theory}

\section{Two-level systems}

\section{Emission and absorption of radiation}

\section{Spontaneous emission}





\end{document}
