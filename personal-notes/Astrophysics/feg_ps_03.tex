\documentclass{_mypackages/monograph}

\title{Formation and Evolution of Galaxies \\ Problem Set 03 \\ Dark Matter} % \MyTitle
\author{Bruno Murino - 8944901} % \MyAuthor
\date{\today} % \MyDate

\addbibresource{qinfo.bib}
\graphicspath{ {figures/} }

\begin{document}
% \frontmatter

\solutionstp
% \dominitoc
% \doparttoc
% \pagestyle{onlypagenum}
% \tableofcontents
% \mainmatter

\subsubsection{1.}

We have that
\begin{equation}
    \eta_{DM}= \frac{\rho_{DM}}{m} = \Omega_{DM} \,\, \rho_{crit,0} \,\, \frac{1}{m}.
\end{equation}

Assuming
\begin{equation}
    \Omega_{DM} \sim 0.258,
\end{equation}
\begin{equation}
    \rho_{crit,0} \approx 10^{-29} \si{\gram\per\cubic\centi\metre},
\end{equation}
and that the mass \(m\) of a WIMP is
\begin{equation}
    m = 100 \si{\giga\electronvolt} = 179\cross 10^{-24} \si{\gram},
\end{equation}
we find that the number density of WIMP's in the universe should be
\begin{equation}
    \eta_{DM} \approx 1.44134 \cross 10^{-8} \si{\per\cubic\centi\metre},
\end{equation}
which is really sm









\subsubsection{2.}



If \(\Omega_m = 1\), then \(\rho_m = \rho_{crit}\) and \(M/L_B\) becomes the critical mass-to-light ratio
\begin{equation}
    \left(\frac{M}{L}\right)_{B,m} = \frac{\rho_m}{\mathcal{L}_B} = \frac{\rho_{crit}}{\mathcal{L}_B} \approx 1500 h \left( \si{\solarmass\per\solarluminosity}\right)_{B},
\end{equation}
where the formula used was equation 2.70 of "Galaxy Formation and Evolution", by Mo, Houjun; Van den Bosch, Frank; and White, Simon





\subsubsection{3.}

We know that for the Coma cluster
\begin{equation}
\begin{split}
    M_* &= 1.0 \cross 10^{13} h^{-1} \si{\solarmass}, \\ 
    M_{gas} &= 5.5 \cross 10^{13} h^{-3/2} \si{\solarmass}, \\ 
    M_T &= 68 \cross 10^{13} h^{-1} \si{\solarmass}.
\end{split}
\end{equation}

Assuming that all baryonic mass is either in the form of stars or gas, then we can find the fraction of baryons in the Coma cluster considering:
\begin{equation}
    \frac{M_b}{M_T} = \frac{M_* + M_{gas}}{M_T} = 0.111379,
\end{equation}
where we used \(h=0.7\). This number means that \(\sim 90\%\) of the Coma cluster mass is in the form of dark matter.



\subsubsection{4.}

In MOND we have that
\begin{equation}
    F = m \mu\left( \frac{a}{a_0}\right) a,
\end{equation}
with \(\mu(x) \to 1\) for \(x\gg 1\) and \(\mu(x) \to x\) for \(x\ll 1\), where \(a_0\) is some undetermined constant, which later was measured to be \(a_0 \approx 1.2 \cross 10^{-8}\si{\centi\metre\per\second}\). If we impose \(a/a_0 \ll 1\), when \(a = v^2/r\) is the centripetal acceleration of the components of a galaxy, we obtain
\begin{equation}
    F = \frac{G Mm}{r^2} = m \frac{\nicefrac{v^2}{r}}{a_0} \frac{v^2}{r} = m \frac{1}{a_0} \frac{v^4}{r^2},
\end{equation}
and then
\begin{equation}
    v = \sqrt[4]{a_0GM},
\end{equation}
implying that the rotation curves are flat, i.e. do not depend on the distance \(r\) from the centre of a galaxy.


\subsubsection{5.}

A density profile described by a Singular Isothermal Sphere is a density such that
\begin{equation}
    \rho(r) = \frac{\sigma^2}{2\pi G r^2}
\end{equation}







% \backmatter
% \printbib
\end{document}