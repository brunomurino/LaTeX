\documentclass[oneside, 12pt]{book}

\usepackage{mypreamble}
\usepackage[backend=biber,style=nature]{biblatex}
%\cite{barton} = [1] and \footfullcite[p. 324]{barton} = ^1 + footnote, p. 324 (e.g.)

\usepackage{mycommands}
\usepackage{mytheme1}

\addbibresource{ref_ST.bib}

\begin{document}

\chapter{Quantization}

\section{Point particle}

The Lagrangian for a massive point particle is
\beq[eq:mfplagrangian] \mc{L} = -\sqrt{-\dot{x}^2} \eeq
which yields the equation of motion
\beq[eq:mfpeom] \dv{p^{\mu}}{\tau} = 0 \eeq
where \(p^{\mu}\) is given by
\beq[eq:mfpmomentum] p^{\mu} = \pdv{\mc{L}}{\tau} = \frac{m\dot{x}^{\mu}}{\sqrt{-\dot{x}^2}} \eeq

But if we impose the light-cone gauge condition
\beq[eq:mfplcgauge] x^{+} = \frac{1}{m^2}p^{+}\tau \eeq
then equation \eqref{eq:mfpmomentum} yields
\beq[eq:mfpmomentum2] p^{\mu} = m^2\dot{x}^{\mu} \eeq
which results in the simplified equation of motion
\beq[eq:mfpeom2] \ddot{x}^{\mu}=0 \eeq

Now, expanding the constrain \eqref{eq:p2m20constrain} in light-cone coordinates we have
\beq[eq:p2m20lcconstrain] p^{-} = \frac{1}{2p^{+}}\left( p^Ip^I + m^2 \right) \eeq
therefore if we know \(p^{+}\) and \(p^{I}\) then we know \(p^{-}\). If we set \(\mu = -,I \) in \eqref{eq:mfpmomentum2}, then we get
\beq[eq:mfpx-] x^{-}(\tau) = x^{-}_0 + \frac{p^{-}}{m^2}\tau \eeq
and
\beq[eq:mfpxI] x^I(\tau) = x^{I}_0 + \frac{p^I}{m^2}\tau  \eeq

Now that we've solved the equations, we know that our dynamical variables are 
\begin{tcolorbox}
    \beq[eq:mfpdynvar] x^I \qc x^{-}_0 \qc p^I \qq{and} p^{+} \eeq
\end{tcolorbox}

\section{Quantum point particle}

In Schrödinger picture, to quantize the point particle we promote the dynamical variables \eqref{eq:mfpdynvar} to time-independent Schrödinger operators and impose the following commutation relations
\beq[eq:mfpcomm] \comm{x^I}{x^J} = i\eta^{IJ} \qc \comm{x^{-}_0}{p^{+}} = i\eta^{-+} = -1 \eeq
The additional Schrödinger operators, which are constructed from the set of time-independent Schrödinger operators and time (\(\tau\)), are \(x^{+}(\tau)\), \(x^{-}(\tau)\) and \(p^{-}\) and are defined using the quantum analogs of equations \eqref{eq:mfplcgauge}, \eqref{eq:mfpx-} and \eqref{eq:p2m20lcconstrain}.

Energy generates time translation. In our case we have
\beq[eq:deltaudelx+] \pdv{\tau} = \frac{p^{+}}{m^2}\pdv{x^{+}} \longleftrightarrow \frac{p^{+}}{m^2}p^{-} \eeq
therefore we postulate the Heisenberg Hamiltonian
\beq[] H(\tau) = \frac{p^{+}(\tau)}{m^2} \eeq

Recall that given a Schrödinger operator \(O_S(\tau)\), to find the time-dependence of the corresponding Heisenberg operator \(O_H(\tau)\) we use
\beq[eq:heisenschrotimerelation] i\dv{O_H}{\tau} = i\pdv{O_S}{\tau} + \comm{O_S}{H_H} \eeq
where \(H_H(\tau)\) is the Heisenberg Hamiltonian associated with the Schrödinger Hamiltonian \(H_S(\tau)\).\par 
Noting that \(H(\tau)\) has no explicit time-dependence, from \eqref{eq:heisenschrotimerelation} we can conclude that it is time independent, and as a consequence we see that \(p^I\) is also time independent. Also, we can check that this Hamiltonian generates the expected equations of motion.\par 
From the list of Schrödinger operators and their commutation relations \eqref{eq:mfpcomm}, we see that we can label the state of a particle using the eigenvalues of \(p^{+}\) and \(p^I\)
\beq[] \ket{p^{+},p^I} \eeq

There is a natural identification of the quantum states of a relativistic point particle of mass \(m\) with the one-particle states of the quantum theory of a scalar field of mass \(m\)
\beq[] \ket{p^{+},\vb{p}_T} \longleftrightarrow a^{\dag}_{p^{+},p_T}\ket{\Omega} \eeq

\section{Quantum open string}

Similarly to what we did to quantize the point particle, we now promote the string dynamical variables \eqref{eq:stringdynvar2} to time-independent Schrödinger operators
\beq[eq:stringschrodop] X^I(\sigma) \qc x^{-}_0 \qc \mc{P}^{\tau I}(\sigma) \qq{and} p^{+} \eeq
with their corresponding Heisenberg operators. Now we must set the commutation relations.\par 
First, we expect measurements to interfere with each other only if they are taken at the same point on the string, so we set
\beq[] \comm{X^I(\sigma)}{\mc{P}^{\tau J}(\sigma')} = i\eta^{IJ}\delta(\sigma - \sigma') \eeq
also, we set
\beq[] \comm{X^I(\sigma)}{X^J(\sigma')} = \comm{\mc{P}^{\tau I}(\sigma)}{\mc{P}^{\tau J}(\sigma')} = 0 \eeq
and
\beq[] \comm{x^{-}_0}{p^{+}} = -i \eeq
the operators \( x^{-}_0\) and \(p^+\) commute with all othe Schrödinger operators \( X^I\) and \( \mc{P}^{\tau I}\).\par 
The associated Heisenberg operators all commute except
\beq[] \comm{X^I(\tau,\sigma)}{\mc{P}^{\tau J}(\tau,\sigma')} = i\eta^{IJ}\delta(\sigma -\sigma') \eeq
and
\beq[] \comm{x^{-}_0(\tau)}{p^{+}(\tau)}=-i \eeq\par
In analogy with \eqref{eq:deltaudelx+}, we have
\beq[eq:deltaudelX+] \pdv{\tau}=2\alpha 'p^{+}\pdv{X^{+}} \longleftrightarrow 2\alpha ' p^{+}\mc{P}^{\tau -} \eeq
and since we have a momentum density on the right side of \eqref{eq:deltaudelX+}, we postulate the following Hamiltonian
\beq[eq:stringhamiltonianp] H = 2\alpha 'p^{+}\int_0^{\pi} \dd{\sigma} \mc{P}^{\tau - } = 2\alpha ' p^{+}p^{-} \eeq
but using \eqref{eq:alphamomentum} and \eqref{eq:stringLmode}, we have
\beq[eq:stringhamiltonian] H = L_0^{\perp} \eeq
We can also expand\par 
Now we can use \eqref{eq:heisenschrotimerelation} and find that \(x^{-}_0\) and \( p^{+}\) are time-independent since they commute with the Hamiltonian, which is also time-independent. Also, using \eqref{eq:heisenschrotimerelation}, this Hamiltonian yields the expected equations of motion \eqref{eq:stringwaveeom}.\par 

From manipulating the commutation relations and using \eqref{eq:stringsol} we can find that
\beq[eq:annicreatcomm] \comm{\alpha_m^I}{\alpha^J_n} = m\eta^{IJ}\delta_{m+n,0} \eeq
and also
\beq[] \comm{x^I_0}{p^J} = i\eta^{IJ} \eeq
Defining
\beq[eq:alpha_a_def] \alpha_n^I = a^I_n \sqrt{n} \qq{and} \alpha^I_{-n}=a_n^{I\dag}\sqrt{n} \qc n\geq 1\eeq
we can find that
\beq[] \comm{a^I_m}{a_n^{J\dag}} = \delta_{m,n}\eta^{IJ} \eeq
which are the commutation relations of the canonical annihilation and creation operators of a quantum simple harmonic oscillator! So we say that \(\alpha^I_n\) are annihilation operators and that \( \alpha^I_{-n}\) are creation operators.\par

The reason we didn't start promoting \eqref{eq:stringdynvar} to operators is that we wouldn't be able to, intuitively, set the commutation relations we just set. But to conclude this, it's important to see that all we need to find \(X^{\mu}\) is \( p^{+}\), to find \(X^{+}\), we need \(x_0^I\), \( p^I\) and every \(a^I_n\) for \(n\geq 1\) to find \(X^I\), and with \(X^I\) we also have \(X^{-}\) up to a constant \(x_0^{-}\).\par 

But lets look at \eqref{eq:stringhamiltonian} again. Recall that, using \eqref{eq:traversevirasoromode_n}, the string Hamiltonian is
\beq[] L_0^{\perp} = \hlf\sum_{p\in \mb{Z}}\alpha^I_{-p}\alpha^I_{p} \eeq
Note that \(p\) can take on negative values, what would make the right-most operator a creation operator, whereas to keep normal-ordering we should have the right-most operator be an annihilation operator instead. So to fix that we split our sum into \(3\) parts, one for \(p=0\), one for the positive \(p\) and other for the negative \(p\)
\beq[eq:L0original] L_0^{\perp} = \hlf \alpha_0^I\alpha_0^I + \hlf \sum_{p=1}^{\infty} \alpha^I_{-p}\alpha^I_{p} + \hlf\sum_{p=1}^{\infty}\alpha^I_{p}\alpha^I_{-p} \eeq
The first term is normal-ordered, and so is the first sum, but the second sum is not. So we do
\beq[] \hlf \sum_{p=1}^{\infty} \alpha_p^I\alpha^I_{-p} = \hlf\sum_{p=1}^{\infty}\left(\alpha^I_{-p}\alpha^I_{p} + \comm{\alpha^I_p}{\alpha^I_{-p}} \right) \eeq
and using \eqref{eq:annicreatcomm} we have
\beq[] \hlf \sum_{p=1}^{\infty} \alpha^I_p \alpha^I_{-p} = \hlf \sum_{p=1}^{I} \alpha^I_{-p}\alpha^I_{p} + \hlf(D-2)\sum_{p=1}^{\infty} p \eeq
Lets define \(a\) as
\beq[] a = \hlf(D-2)\sum_{p=1}^{\infty}p \eeq
And now lets define \(L_0^{\perp}\) to be the normal-ordered operator \eqref{eq:L0original} with out the ordering constant \(a\). Using \eqref{eq:alphamomentum} and \eqref{eq:alpha_a_def} we can write
\beq[] L_0^{\perp} \equiv \hlf\alpha_0^I\alpha_0^I + \sum_{p=1}^{\infty} \alpha^I_{-p}\alpha^I_{p} = \alpha'p^Ip^I + \sum_{p=1}^{\infty}p a_p^{I\dag}a_p^I \eeq
With this new definition of \(L_0^{\perp}\), \eqref{eq:L0p+p-} becomes
\beq[] 2\alpha'p^{-} \equiv \frac{1}{p^{+}}\left(L_0^{\perp} + a \right) \eeq
and if we use plug this into \eqref{eq:p2m20constrain} we have
\beq[] M^2 = -p^2 = 2p^{+}p^{-} - p^Ip^I = \frac{1}{\alpha'}\left( L_0^{\perp} + a - \alpha' p^Ip^I\right) = \frac{1}{\alpha'}\left(a + \sum_{n=1}^{\infty}na_n^{I\dag}a^I_n \right)  \eeq
which is slightly different from \eqref{eq:stringclassicalmass}. Of course, the conjugated sign (\(^*\)) now became the hermitian conjugated sign (\(^{\dag}\)) since we are dealing with operators instead of numbers. But the real difference is the presence of the ordering constant \(a\), which is a consequence of the non-zero commutator between annihilation and creation operators, which in the classical case is always zero.\par 

\section{Quantum closed string}

\section{Supersymmetry}




\end{document}