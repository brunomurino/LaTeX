\documentclass{_mypackages/monograph}

\title{Formation and Evolution of Galaxies \\ Problem Set 06 \\ Distant Galaxies} % \MyTitle
\author{Bruno Murino - 8944901} % \MyAuthor
\date{\today} % \MyDate

\addbibresource{qinfo.bib}
\graphicspath{ {figures/} }

\begin{document}
% \frontmatter

\solutionstp
% \dominitoc
% \doparttoc
% \pagestyle{onlypagenum}
% \tableofcontents
% \mainmatter

\subsubsection{1)}

The optical wavelengths corresponds to the interval from \(\lambda_i = 390\cross 10^{-9}m\) up to \(\lambda_f=700\cross 10^{-9}m\). From the definition of redshift we know that
\begin{equation}
    z = \frac{\lambda_{obs}}{\lambda_{em}} - 1.
\end{equation}
Since the H-\(\alpha\) spectral line corresponds to \(\lambda_{em} = 656.28\cross 10^{-9}m\), if we want to observe it on the visible region, we need that
\begin{equation}
    z_i = \frac{\lambda_i}{\lambda_{em}} - 1 = \frac{390}{656.28} - 1 \approx -0.405741,
\end{equation}
and
\begin{equation}
    z_f = \frac{\lambda_i}{\lambda_{em}} - 1 = \frac{700}{656.28} - 1 \approx 0.0666179.
\end{equation}



\subsubsection{2)}

The NIR corresponds to the interval from \(\lambda_i=700\cross 10^{-9}m\) up to \(\lambda_f = 10^{-3}m\). From the definition of redshift we know that
\begin{equation}
    z = \frac{\lambda_{obs}}{\lambda_{em}} - 1.
\end{equation}
Since the Lyman-\(\alpha\) spectral line corresponds to \(\lambda_{em} = 121.6\cross 10^{-9}m\), if we want to observe it on the NIR region, we need that
\begin{equation}
    z_i = \frac{\lambda_i}{\lambda_{em}} - 1 = \frac{700}{121.6} - 1 \approx 4.76,
\end{equation}
and
\begin{equation}
    z_f = \frac{\lambda_i}{\lambda_{em}} - 1 = \frac{10^6}{121.6} - 1 \approx 8223.
\end{equation}










% \backmatter
% \printbib
\end{document}