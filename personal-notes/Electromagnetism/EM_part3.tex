\documentclass[oneside, 12pt, notitlepage]{book}

\usepackage{../_mypackages/mypreamble}
\usepackage{../_mypackages/mycommands}
\usepackage{../_mythemes/notestheme}

\addbibresource{EM_ref.bib}

%----------------------------------END PREAMBLE---------------------------------

\begin{document}

\frontmatter
\pagestyle{empty}
\notestp{Electrodynamics}
\begin{notesabstract}[Abstract]
Abstract text
\end{notesabstract}
\centeredtoc{Content}
\mainmatter
\pagestyle{mynotespage} %\normalfont

\chapter{Introduction}

\begin{tcolorbox}
\beq[eq:maxwelleq]
    \left.\begin{aligned}
        \div{\vb{E}} &= \frac{\rho}{\epsilon_0}\\
        \div{\vb{B}} &= 0\\
        \curl{\vb{E}} +\pdv{\vb{B}}{t} &= 0\\
        \curl{\vb{B}} - \mu_0 \epsilon_0\pdv{\vb{E}}{t} &= \mu_0\vb{J}
       \end{aligned}
    \right\}
    \qquad \text{Maxwell's equations}
\eeq
\end{tcolorbox}

\begin{tcolorbox}
\beq[eq:maxwelleqboundary]
    \left.\begin{aligned}
    \vb{E}^{\perp}_{above} - \vb{E}^{\perp}_{under} &= \frac{\sigma}{\epsilon_0}\\
    \vb{E}^{\parallel}_{above} - \vb{E}^{\parallel}_{under} &= 0\\
    \vb{B}^{\perp}_{above} - \vb{B}^{\perp}_{under} &= 0\\
    \vb{B}^{\parallel}_{above} - \vb{B}^{\parallel}_{under} &= \mu_0\vb{K}\cross \vu{n}
    \end{aligned}
\right\}
\qquad \text{Boundary conditions}
\eeq
\end{tcolorbox}

\chapter{Wave guides}

Lets study monochromatic plane waves inside wave guides.\par

Assume a plane wave propagating along the \(z\) axis, then
\beq[] \vb{E} = \tvb{E}(x,y) \exp{i\lr{k_zz - \omega t}} \eeq
\beq[] \vb{B} = \tvb{B}(x,y) \exp{i\lr{k_zz - \omega t}} \eeq
Plugging these into Maxwell's equations \eqref{eq:maxwelleq} we find
\begin{tcolorbox}
\beq[]
    \begin{aligned}
        \frac{i}{(\omega/c)^2-k_z^2} \lr{ k\pdv{E_z}{x} + \omega \pdv{B_z}{y} } &= E_x \\
        \frac{i}{(\omega/c)^2-k_z^2} \lr{ k\pdv{E_z}{y} - \omega \pdv{B_z}{x} } &= E_y\\
        \frac{i}{(\omega/c)^2-k_z^2} \lr{ k\pdv{B_z}{x} - \frac{\omega}{c^2} \pdv{E_z}{y} } &= B_x\\
        \frac{i}{(\omega/c)^2-k_z^2} \lr{ k\pdv{B_z}{y} + \frac{\omega}{c^2} \pdv{E_z}{x} } &= B_y
       \end{aligned}
\eeq
\end{tcolorbox}
with
\begin{tcolorbox}
\beq[eq:ezbzeq]
    \begin{aligned}
        \left[ \pdv[2]{x} + \pdv[2]{y} -k_z^2+ \left(\frac{\omega}{c} \right)^2 \right]E_z &= 0\\
        \left[ \pdv[2]{x} + \pdv[2]{y} -k_z^2+ \left(\frac{\omega}{c} \right)^2 \right]B_y &=0
       \end{aligned}
\eeq
\end{tcolorbox}\par

This means that in order to fully find \(\tvb{E}\) and \(\tvb{B}\) we need need just \(E_z\) and \(B_z\), and we find them solving \eqref{eq:ezbzeq}.\par

\section{The Coaxial Transmission Line}

The Laplacian in cylindrical coordinates is
\beq[] \laplacian = \frac{1}{\rho} \pdv{\rho}\lr{\rho \pdv{\rho}} + \frac{1}{\rho^2}\pdv[2]{\phi} + \pdv[2]{z} \eeq\par

Our ansatz for \(E_z\) is
\beq[] E_z = E_0 R(\rho) \exp{im\phi}\exp{i\lr{k_z z - \omega t}}\qc m = 0,\pm 1, \pm 2,... \eeq\par

Doing
\beq[] s = \rho \sqrt{\lr{\frac{\omega}{c}}^2 - k_z^2} \eeq
we find the equation for \(R(\rho)\), which is \textit{Bessel's differential equation}
\beq[] s^2 R'' + sR' + (s^2 - m^2)R=0 \eeq

\chapter{Resonant Cavities}

\chapter{Scattering and Diffraction}

\section{Scalar wave theory}

We could do this the normal way, using the fields as vectors and etc. However, we get roughly the \textit{same} results if we use a scalar theory, which is much easier to work with.\par

In electrodynamics we know that \(\vb{E}\) and \(\vb{B}\) satisfy:
\beq[]
    \left(\laplacian - \frac{1}{c^2}\pdv[2]{t} \right)
    \left\{\begin{aligned}
        \vb{E}\\
        \vb{B}
       \end{aligned}
       \right\} = 0
\qquad \text{Boundary conditions}
\eeq
So in this scalar theory we make the scalar field \(\psi\) satisfy
\beq[eq:scalarwe] \left(\laplacian - \frac{1}{c^2}\pdv[2]{t} \right)\psi = 0 \eeq
but what are the solution to this equation \eqref{eq:scalarwe}? Of course, the simplest solution is the plane wave
\beq[] \psi = A\exp{\vb{k}\cdot\vb{r} -i\omega t} \eeq
To simplify our notation, lets write \(\psi = \psi_w \exp{-i\omega t}\) and omit \(\exp{-i\omega t}\), but remember that it is always there!\par

We can also solve \eqref{eq:scalarwe} in spherical coordinates, then our general solution is
\beq[eq:psisphe] \psi_w = \sum_{l=0}^{\infty}\sum_{m = -l}^{l}\lr{A_{lm}j_l(kr) + B_{lm}n_l(kr)}Y_{lm}\lr{\theta,\phi} \eeq
or alternatively
\beq[] \psi_w = \sum_{l=0}^{\infty}\sum_{m = -l}^{l}c_l\lr{\cos(\delta_l) j_l(kr) - \sin(\delta_l)n_l(kr)}Y_{lm}\lr{\theta,\phi}\eeq
where
\beq[] j_l(x) = \sqrt{\frac{\pi}{2x}}J_{l+\hlf}(x) = (-x)^l\lr{\frac{1}{x}\pdv{x}}^l \lr{\frac{\sin(x)}{x}} \eeq
\beq[] n_l(x) = \sqrt{\frac{\pi}{2x}} N_{l+\hlf}(x) = - (-x)^l\lr{\frac{1}{x}\pdv{x}}^l \lr{\frac{\cos(x)}{x}} \eeq
with \(J\) being Bessel functions and \(N\) being Neumann functions, and \(Y\) being the spherical harmonics.\par

\begin{tcolorbox}
\beq[eq:krgtl]
    x = kr >> l \Rightarrow
    \left\{\begin{aligned}
        j_l &\approx \frac{1}{x}\sin\lr{x - \frac{l\pi}{2}} \\
        n_l &\approx -\frac{1}{x}\cos\lr{x - \frac{l\pi}{2}}
       \end{aligned}
       \right.
\eeq
\beq[eq:krstl]
    x = kr << l \Rightarrow
    \left\{\begin{aligned}
        j_l &\approx \frac{x^l}{(2l+1)!!}\lr{1 - \frac{x^2}{2(2l+3)}+...} \\
        n_l &\approx -\frac{(2l+1)!!}{x^{l+1}}\lr{1 - \frac{x^2}{2(1-2l)}+...}
       \end{aligned}
       \right.
\eeq
\end{tcolorbox}

It's very important to notice that this is only a choice of basis, meaning that we \textbf{\textit{can}} write even a plane wave in the form \eqref{eq:psisphe}. This is actually called the \textit{Rayleigh expansion} and it reads
\beq[] \exp{i\vb{k}\cdot\vb{r}} = 4\pi \sum_{lm} i^l j_l(kr)Y^{*}_{lm}(\vu{k})Y_{lm}(\vu{r}) \eeq
and in particular, if \( \vu{k} = \vu{z}\) with \( \vu{k}\cdot\vu{r} = \cos \theta \), then
\beq[] \exp{i\vb{k}\cdot\vb{r}} = \exp{ikr\cos \theta} = 4\pi \sum_l i^l j_l(kr)
\underbrace{\sum_m Y^{*}_{lm}(\vu{z})Y_{lm}(\vu{r})}_{\frac{(2l+1)}{4\pi}P_l(\cos \theta)} \eeq
so
\beq[] \exp{ikr\cos \theta} =  \sum_l i^l (2l+1) j_l(kr) P_l(\cos \theta) \eeq
which also actually proves Huygens theorem: we can interpret the statement that when a wave reaches an object, every point becomes a secondary source of a spherical wave as just a change of basis, a change of the \textit{way} we are describing the same wave.\par

The scattering we'll be studying is the one from a plane wave with an axially symmetric object, so the scattered wave doesn't depend on \(\phi\). Then
\beq[] \psi_{in} = A\exp{ikz} \eeq
Then our total scalar wave is
\beq[] \psi_T = \psi_{in} + \psi_{s} \eeq
\begin{tcolorbox}[colback=blue!50!white]
\beq[eq:totalpsi] \psi_T = A\sum_l c_l \lr{\cos\delta_l j_l(kr) - \sin\delta_l n_l(kr)}P_l(\cos \theta) \eeq
\end{tcolorbox}
Also, we'll demand that
\beq[] \psi_{scattered} \rightarrow A \frac{\exp{i(kr - wt)}}{r}f(\theta,\phi) \eeq
far away. Upon such demand, we find that
\beq[] c_l = i^l (2l+1)\exp{i\delta_l} \eeq
and as a consequence
\beq[eq:angularf] f(\theta) = \frac{1}{k} \sum_{l=0}^{\infty} \exp{i\delta_l}(2l+1)\sin\delta_l P_l(\cos\theta) \eeq\par

We can, then, impose either Dirichlet boundary conditions or Neumann boundary conditions on \eqref{eq:totalpsi} to find the \(\delta_l\), which we'll put back on \eqref{eq:angularf} and then be able to study the experimental data using the cross section, which we'll see on the next section.\par

\section{Cross section}

The emitted power per solid angle of a source is
\beq[] \mean{\dv{P_e}{\Omega}} \eeq
The incident flux is
\beq[] \mean{\abs{\vb{S}}} \eeq
where \(\vb{S}\) is the Poynting vector.\par
The cross section is defined as
\beq[] \dv{\sigma}{\Omega} = \frac{\mean{\dv{P_e}{\Omega}}}{\mean{\abs{\vb{S}}}} \eeq

In this scalar theory
\beq[] \psi_{in} = A\exp{ikz} \rightarrow \abs{\vb{S}}^2 = A^2 \eeq
and
\beq[] \underbrace{\psi_e \approx A \frac{\exp{ikr}}{r}f(\theta,\phi)}_{r\rightarrow \infty}  \rightarrow \mean{\dv{P_e}{\Omega}} = A^2 \abs{f(\theta,\phi)}^2 \eeq
so our differential cross section is
\beq[] \dv{\sigma}{\Omega} = \abs{f(\theta,\phi)}^2 \eeq
and the full cross section is
\beq[] \sigma = \int \dd{\Omega} \abs{f(\theta,\phi)}^2 \eeq\par

To shed light into some computations, lets calculate \(\abs{f(\theta,\phi)}^2\) recalling \eqref{eq:angularf}:
\beq[] \abs{f(\theta,\phi)}^2 =  \frac{1}{k^2}\sum_{l}\exp{i\delta_l}(2l+1)\sin\delta_l P_l(\cos\theta)\sum_{l'}\exp{-i\delta_{l'}}(2l'+1)\sin\delta_{l'} P_{l'}(\cos\theta) \eeq
\beq[] \abs{f(\theta,\phi)}^2 =  \frac{1}{k^2}\sum_{l}\sum_{l'}\exp{i\left(\delta_l-\delta_{l'}\right)}(2l+1)(2l'+1)\sin\delta_l \sin \delta_{l'} P_l(\cos\theta)P_{l'}(\cos\theta) \eeq
Now we must integrate it to find the cross section \(\sigma\)
\beq[] \sigma = \int \dd{\Omega} \frac{1}{k^2}\sum_{l}\sum_{l'}\exp{i\left(\delta_l-\delta_{l'}\right)}(2l+1)(2l'+1)\sin\delta_l \sin \delta_{l'} P_l(\cos\theta)P_{l'}(\cos\theta) \eeq
\beq[] \sigma = \frac{1}{k^2}\sum_{l}\sum_{l'}\exp{i\left(\delta_l-\delta_{l'}\right)}(2l+1)(2l'+1)\sin\delta_l \sin \delta_{l'}\underbrace{ \int \dd{\Omega} P_l(\cos\theta)P_{l'}(\cos\theta)}_{\frac{4\pi}{(2l + 1)}\delta_{ll'}} \eeq
\beq[] \sigma = \frac{1}{k^2}\sum_{l}(2l+1)^2\sin^2 \delta_l \frac{4\pi}{(2l + 1)} \eeq
\begin{tcolorbox}[colback=blue!50!white]
\beq[eq:finalcrosssection] \sigma = \frac{4\pi}{k^2}\sum_{l}(2l+1)\sin^2 \delta_l \eeq
\end{tcolorbox}
And this is the final expression for our cross section.\par

\section{Perfectly conducting sphere}

If our object is a perfectly conducting sphere of radius \(a\), then we must impose Dirichlet conditions on \eqref{eq:totalpsi} at \(r=a\). If we do that we find that
\beq[] \tan\delta_l = \frac{j_l(ka)}{n_l(ka)} \eeq\par

Now is a good place to split our study in two cases:
\begin{tcolorbox}
\beq[] \qq{when} ka >> l \qq{} \rightarrow \qq{Mie scattering} \eeq
\beq[] \qq{when} ka << l \qq{} \rightarrow \qq{Rayleigh scattering} \eeq
\end{tcolorbox}

\section{Rayleigh scattering}

Using \eqref{eq:krstl} we find that
\beq[] \tan\delta_l \approx - \frac{(ka)^{2l+1}}{\left[(2l+1)!!\right]^2} << 1 \eeq
then
\beq[] \tan\delta_l \approx \delta_l \approx - \frac{(ka)^{2l+1}}{\left[(2l+1)!!\right]^2} \eeq\par

Lets compute the first two \(\delta\)
\beq[] \delta_0 = -ka \qcomma \delta_1 = -\frac{(ka)^3}{9} \eeq
Then, since \(\sin \delta_l \approx \delta_l\), the cross section is, up to the second order (\(l=1\)),
\beq[] \sigma = \underbrace{4\pi a^2}_{l=0} + \underbrace{\frac{4\pi k^4 a}{27}}_{l=1} \eeq\par

If we consider this Rayleigh scattering in the context of electrodynamics, then we must remember that there is no radiation from a monopole, this means that if we had studied this scattering using vector fields, every term relative to \(l=0\) would vanish! So to achieve a proper parallel between the scalar theory and the vector theory, we must, by hand, eliminate the \(l=0\) terms, so the cross section up to the second order (\(l=1\)) of a Rayleigh scattering by a perfectly conductor sphere of radius \(a\) is
\beq[] \sigma = \frac{4\pi}{27}k^4 a \eeq

\section{Mie scattering}

Using \eqref{eq:krgtl} we find that
\beq[] \tan \delta_l = \tan\left[-\lr{ka - \frac{l\pi}{2}}\right] \eeq
so
\beq[] \delta_l = -\lr{ka - \frac{l\pi}{2}} \eeq
and since we want to plug this into \eqref{eq:finalcrosssection} we must compute \(\sin^2 \delta_l\)
\beq[] \sin^2 \delta_l = \sin^2\lr{ka - \frac{l\pi}{2}} =  \eeq
\beq[] \sin^2\lr{ka - \frac{l\pi}{2}} = \left[\sin (ka) \cos\lr{ \frac{l\pi}{2}} - \sin \lr{\frac{l\pi}{2}}\cos (ka)\right]^2 \eeq
and recalling that \(l\) is an integer:
\beq[] \left[\sin (ka) \cos\lr{ \frac{l\pi}{2}} - \sin \lr{\frac{l\pi}{2}}\cos (ka)\right]^2 = \eeq
\beq[] = \sin^2 (ka) \cos^2 \lr{\frac{l\pi}{2}} + \sin^2 \lr{\frac{l\pi}{2}}\cos^2 (ka) \eeq
then the cross section becomes
\beq[eq:sigmastep1] \sigma = \frac{4\pi}{k^2} \{ \sin^2 ka \underbrace{\sum_l (2l+1)\cos^2 \frac{l\pi}{2}}_{\approx \frac{(ka)^2}{2}} + \cos^2 ka \underbrace{\sum_l (2l+1)\sin^2\frac{l\pi}{2}}_{\approx \frac{(ka)^2}{2}} \} \eeq
the approximations comes from the fact that \(l << ka\), so the max acceptable value for \(l\) is \(ka\), so the sum in \eqref{eq:sigmastep1} is limited to \(l=ka\), thus giving the result
\beq[] \sigma = 2\pi a^2 \eeq


\chapter{Diffraction}

Let the scalar field be \(\psi(\vb{x},t)\) satisfying the Helmholtz wave equation
\eq{\lr{\laplacian + k^2}\psi(\vb{x})=0}
and let it have harmonic time dependence \(\exp{-i\omega t}\). Let \(\phi(\vb{x},\vb{x}')\) be a Green function of the Helmholtz wave equation
\eq{\lr{\laplacian + k^2}\phi(\vb{x},\vb{x}') = -\delta(\vb{x} - \vb{x}')}
In Green's theorem
\eq[eq:greenstheorem]{\uint[V] \dd[3]{x'} \left[ f \laplacian{g} - g \laplacian{f}\right] = -\uoint[\del V] \dd{\vb{S'}} \left[ f \grad{g} - g\grad{f}\right]}
by letting \(f = \phi(\vb{x},\vb{x}')\) and \(g = \psi(\vb{x}')\) we find that
\eq{\psi(\vb{x}) = \uoint[\del V] \dd{S'} \left[ \psi(\vb{x}')\vu{n}'\cdot \grad_{x'}\phi - \phi \vu{n}'\cdot \grad_{x'}{\psi} \right] }
where \(\vu{n}'\) is an inwardly directed normal to the surface \(\del V \equiv S\).\par

Let
\eq{\phi(\vb{x},\vb{x}') = \frac{\exp{ikR}}{4\pi R} }
where \(\vb{R} = \vb{x}-\vb{x}'\) and
\eq{\psi \rightarrow f(\theta,\phi) \frac{\exp{ikr}}{r} \qcomma \frac{1}{\psi}\pdv{\psi}{r} \rightarrow \lr{ik - \frac{1}{r}}}\par

In the context of diffraction, let \(S = S_1 + S_2\), where \(S_1\) is the surface containing the screen and its apertures and \(S_2\) is the surface that goes to infinity. Given the nature of \(\psi\) at \(S\) its easy to see that the integral over \(S_2\) won't contribute to \(\psi(\vb{x})\). Given this, the resulting integral is called \emph{Kirchhoff integral} and reads
\eq{\psi(\vb{x}) = -\frac{1}{4\pi}\uint[S_1] \dd{S} \frac{\exp{ikR}}{R}\vu{n}'\cdot \left[  \grad'\psi + ik\lr{1 + \frac{i}{kR}}\frac{\vb{R}}{R}\psi   \right]}
To compute such integral its necessary to know \(\psi\) and \(\pdv*{\psi}{n}\) on the surface \(S_1\). The \emph{Kirchhoff approximation} consists of the assumptions:
\begin{enumerate}
	\item \(\psi\) and \(\pdv*{\psi}{n}\) vanish everywhere on \(S_1\) except at the apertures.
	\item The values of \(\psi\) and \(\pdv*{\psi}{n}\) at the apertures are equal to the values of the incident wave in the absence of any screen or obstacles.
\end{enumerate}\par
Now the field can be written, for a spherical incident wave, in the common form
\eq{\psi(\vb{R}) = -\frac{ik}{4\pi}A\uint[S_0] \dd{S'} \frac{\exp{ik(R+R_s)}}{RR_s}\vu{n}'\lr{\vu{R}+\vu{R}_s}}
where \(S_0\) are the apertures, and \(\vb{R}_s\) is the position of the source of the incident field.\par

Lets assume that the apertures lie on the \(z=0\) plane, then \(\vu{n}'=\vu{z}\). Let the incident be a plane wave along \(\vu{z}\) such that
\eq{\frac{\exp{ikR}}{R} \rightarrow \exp{ikz} \rightarrow 1 \qq{at the apertures}}
and since the source of a plane wave along \(\vu{z}\) can be considered to be parallel to the \(z=0\) plane, we can write
\eq{\theta_s = 0 \rightarrow \vu{R}_s = \vu{z}}
Lets look for the field \(\psi(\vb{x})\) only at large \(z\), meaning
\eq{\theta = 0 \rightarrow \vu{R} = \vu{z}}
The consequence of such approximations is the so called \emph{Diffraction integral} or \emph{Kirchhoff integral}
\begin{tcolorbox}
\eq{\psi(\vb{x}) &= -\frac{ikA}{2\pi} \uint[S_0] \dd{S'} \frac{\exp{ikR}}{R} }
\end{tcolorbox}
Now let the apertures be small, meaning \(x',y' \ll z\), which implies
\eq{R = \abs{\vb{x}-\vb{x}'} = \sqrt{(x-x')^2 + (y-y')^2 + z^2} \cong z\lr{1 + \hlf \frac{\lr{\vb{x}_{\perp} - \vb{x}_{\perp}'}^2}{z^2} + \cdots}}
where \(\vb{x}_{\perp} = (x,y,0)\) and \(\vb{x}_{\perp}' = (x',y',0)\). Then we have
\eq{\psi(\vb{x}) &= -\frac{ikA}{2\pi} \uint[S_0] \dd{S'} \frac{\exp{ikR}}{R} \\
&= -\frac{ikA}{2\pi} \uint[S_0] \dd{\vb{x}_{\perp}'} \frac{\lexp{ik\lr{z + \hlf \frac{\lr{\vb{x}_{\perp} - \vb{x}_{\perp}'}^2}{z} + \cdots}}}{z + \cdots} \\
&= -\frac{ikA}{2\pi}\frac{\exp{ikz}}{z} \uint[S_0]\dd{\vb{x}_{\perp}'} \lexp{ \frac{ik}{2z}\lr{\vb{x}_{\perp} + \vb{x_{\perp}'}  }^2}}
and we'll usually look to the field on the \(z=z_0\) plane, so
\eq[eq:almostfresnel]{\psi(x,y,z_0) = -\frac{ikA}{2\pi}\frac{\exp{ikz_0}}{z_0} \uint[S_0]\dd{\vb{x}_{\perp}'} \lexp{ \frac{ik}{2z_0}\lr{\vb{x}_{\perp} + \vb{x_{\perp}'}  }^2}}\par

\section{Fresnel diffraction}

In the case of the apertures having cartesian symmetry, its useful to write \(\lr{\vb{x}_{\perp} + \vb{x_{\perp}'}}^2 = (x-x')^2 + (y-y')^2\), with \(x'\) going from \(x^{-}\) to \(x^{+}\), and \(y'\) going from \(y^{-}\) to \(y^{+}\), meaning
\eq{\psi(x,y,z_0) = -\frac{ikA}{2\pi}\frac{\exp{ikz_0}}{z_0} \int_{x^{-}}^{x^{+}} \dd{x'}\lexp{\frac{ik}{2z_0}\lr{x'-x}^2} \int_{y^{-}}^{y^{+}} \dd{y'}\lexp{\frac{ik}{2z_0}\lr{y'-y}^2}}
Letting
\eq{\frac{k}{2z_0}\lr{x'-x}^2 = \frac{\pi}{2}\tau_x'^2 \qq{with} \dd{x'}=\sqrt{\frac{z_0\pi}{k}}\dd{\tau_x'}}
\eq{\sqrt{\frac{k}{z_0\pi}}\lr{x^{+}-x} \equiv u_x^{+} }
\eq{\sqrt{\frac{k}{z_0\pi}}\lr{x^{-}-x} \equiv u_x^{-} }
and the same for \(y'\), then we can write the field as
\eq{\psi(x,y,z_0) = -\frac{ikA}{2\pi} \frac{\exp{ikz_0}}{z_0}\frac{z_0\pi}{k} \int_{u_x^{-}}^{u_x^{+}} \dd{\tau_x'}\lexp{i\frac{\pi}{2}\tau_x'^2} \int_{u_y^{-}}^{u_y^{+}} \dd{\tau_y'}\lexp{i\frac{\pi}{2}\tau_y'^2}     }

\subsubsection{Cornu function}

The Cornu function is
\eq{F(u) = \int_0^u \dd{\tau} \lexp{i\frac{\pi}{2}\tau^2} = \mc{C}(u) + i \mc{S}(u)}
where \(\mc{C}(u)\) is \emph{Fresnel's cossine}, and \(\mc{S}(u)\) is \emph{Fresnel's sine}, both are odd functions, meaning \(F(-u) = -F(u)\).\par

The assymptotic limits are
\eq{\mc{C}(u) \rightarrow u + \mc{O}(u^5) \qq{as} u\rightarrow 0}
\eq{\mc{S}(u) \rightarrow \frac{\pi}{6}u^3 + \mc{O}(u^5) \qq{as} u\rightarrow 0}
\eq{F(u) \rightarrow u + i\frac{\pi}{6}u^3 + \mc{O}(u^5) \qq{as} u\rightarrow 0}
and
\eq{F(u)\rightarrow \hlf (1+i) - \frac{i}{\pi u} \exp{i\frac{\pi}{2}u^2} \qq{as} u \rightarrow \infty}\par

With the Cornu function we can easily write the field \(\psi(x,y,z_0)\) as
\eq{\psi(x,y,z_0) = -\frac{ikA}{2\pi} \frac{\exp{ikz_0}}{z_0}\frac{z_0\pi}{k} \left[F(u_x^{+}) - F(u_x^{-})\right]\left[F(u_y^{+}) - F(u_y^{-})\right]  }
To further simplify our expression for the field, we write
\eq{ u_x^{\pm} = -u_x + \Delta u_x^{\pm} \qq{with} u_x \equiv \sqrt{\frac{k}{z_0 \pi}}x  \qq{and} \Delta u_x^{\pm} \equiv \sqrt{\frac{k}{z_0 \pi}}x^{\pm} }
and the same for \(y\), meaning
\eq{\psi(x,y,z_0) = -\frac{ikA}{2\pi} \frac{\exp{ikz_0}}{z_0}\frac{z_0\pi}{k}&\cross\\
[F(u_x-\Delta u_x^{-}) &- F(u_x - \Delta u_x^{+})]\\
&[F(u_y - \Delta u_y^{-}) - F(u_y - \Delta u_y^{+})]}

\section{Fraunhofer diffraction}

If we expand the expoent in \eqref{eq:almostfresnel} we find
\eq{\psi(\vb{x})= -\frac{ikA}{2\pi}\frac{\lexp{ik\lr{z + \frac{x_{\perp}^2}{z}}}}{z} \uint[S_0]\dd{\vb{x}_{\perp}'} \lexp{ \frac{ik}{2z}\lr{x_{\perp}'^2 - 2\vb{x}_{\perp}\cdot \vb{x}_{\perp}'}}	}
The Fraunhofer limit consists in assuming \(\abs{\vb{x}_{\perp}'}^2 \ll \abs{\vb{x}_{\perp}\cdot \vb{x}_{\perp}'}\), which leads to
\eq{\psi(\vb{x}) = -\frac{ikA}{2\pi}\frac{\lexp{ik\lr{z + x_{\perp}^2/z}}}{z} \uint[S_0]\dd{\vb{x}_{\perp}'} \lexp{- \frac{ik}{z}\lr{\vb{x}_{\perp}\cdot \vb{x}_{\perp}'}}	}
and letting \(\vb{q} = \frac{k}{z}\vb{x}_{\perp}\)
we can write
\begin{tcolorbox} \eq{\psi(\vb{x}) = -\frac{ikA}{2\pi}\frac{\lexp{ik\lr{z + x_{\perp}^2/z}}}{z} \uint[S_0]\dd{\vb{x}_{\perp}'} \exp{- i \vb{q}\cdot \vb{x}_{\perp}'}} \end{tcolorbox}
\noindent which is the diffraction integral considering the so called \emph{Fraunhofer approximation}.\par

Of course, since in the Fresnel diffraction we didn't use the Fraunhofer limit, Fraunhofer diffraction has to be contained in Fresnel diffraction. To verify this, we must understand a bit more about the Fraunhofer limit.\par

In the case of the apertures having cartesian symmetry we can expand the Fraunhofer limit as
\eq{x'^2 + y'^2 \ll \abs{xx' + yy'}}
then
\eq{x'^2 + y'^2 \ll \abs{xx' + yy'} < \abs{xx'} + \abs{yy'}}
This inequality has to hold for all \(x\) and \(y\), so we must ensure the inequality for each component, meaning
\eq{x'^2 \ll \abs{xx'} \qq{and} y'^2 \ll \abs{yy'}}
which implies
\eq{\abs{x'} \ll \abs{x} \qq{and} \abs{y'} \ll \abs{y}}
then
\eq{\abs{x'^{+}},\abs{x'^{-}} \ll \abs{x} \qq{and} \abs{y'^{+}},\abs{y'^{-}} \ll \abs{y}}
So the Fraunhofer limit consists in looking for the field only far away from the aperture area.\par

To conclude, upon the Fraunhofer limit the argument of the Cornu function is huge, so the assymptotic limit \(u\rightarrow \infty\) is aplicable.\par












\backmatter

% \printbib{References}

\end{document}
