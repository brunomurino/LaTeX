\documentclass[oneside, 12pt]{book}

\usepackage{../mypreamble}
% \usepackage[backend=biber,style=nature]{biblatex}
%\cite{barton} = [1] and \footfullcite[p. 324]{barton} = ^1 + footnote, p. 324 (e.g.)

\usepackage{../mycommands}
\usepackage{../mytheme1}

% \addbibresource{ref_ST.bib}

%--------------------------------------------------------------



%--------------------------------------------------------------

\begin{document}

\pagestyle{mypage2} \normalfont

\chapter{Tools to solve the Equations of Motion}

\section{Special Relativity}

When we enter the realm of special relativity it's important to generalise the concept of scalar product. Time and space must be on equal footing now. It can be shown that the appropriate way to generalise the scalar product is to introduce an entity called \textit{metric}, denoted generally by \(g\), a matrix that will tell how two vectors must combine to result in a \textit{scalar}. If the spacetime is not curved, then it is \textit{flat} and the corresponding metric is the famous \textit{Minkowski metric}, which will be denoted \(\eta\) and used with signature \((-1,1,...,1)\). Given two vectors \(a\) and \(b\), their scalar product is defined as
\beq[] a\cdot b = g_{\mu\nu}a^{\mu}b^{\nu} \eeq
where Einstein's summation convention is used (if an index appears twice, the it is summed over).\par

Other important property of the metric \(g\) is to raise/lower indices
\beq[] a_{\mu} = g_{\mu \nu} a^{\nu} \eeq
\beq[] a^{\mu} = g^{\mu \nu} a_{\nu}  \eeq

In relativity is of the utmost importance to find properties that different observers agree on, so a class of transformations has been found, the ones that preserve the relativistic scalar product: the Lorentz transformations.\par

\subsection{Lorentz Transformations}

If we want to preserve a scalar product in flat spacetime we must have that
\beq[] a\cdot b = \eta_{\mu \nu}a^{\mu}b^{\nu}  = \eta_{\mu \nu}a'^{\mu}b'^{\nu} = a'\cdot b' \eeq\par

Let \(\Lambda \) be a general coordinate transformation, then
\beq[] a'^{\mu} = \Lambda^{\mu}_{\nu}a^{\nu}\eeq
therefore we need that
\beq[] \eta_{\mu \nu} = \eta_{\rho \sigma}\Lambda^{\sigma}_ {\ \mu}\Lambda^{\rho}_ {\ \nu}\eeq
or in matrix notation
\beq[] \Lambda^{T} \eta \Lambda = \eta\eeq\par

Such transformation of coordinates is called a \textit{Lorentz transformation} if satisfies the above condition. Also, from the condition, it follows that \( det\Lambda = \pm 1\).\par

\subsection{Some invariant quantities}

Now let's see some important quantities that are preserved under Lorentz transformations.\par

If \(a^{\mu}=b^{\mu}=\text{spacetime coordinates}=x^{\mu}\), then
\beq[] x\cdot x = x^2 \doteq -\Delta S^2 \eeq
where \(\Delta S\) is called \textit{invariant interval} (which I mentioned before!)

Consider two events in spacetime that are \(\dd{x}^{\mu}\) away from each other. Let \(- \dd{s}^2 = \dd{x}_{\mu}\cdot \dd{x}^{\nu} \).
\begin{itemize}
    \item If \( \dd{s}^2 > 0\), then \( \dd{x}^{\mu}\) is said to be a \textit{timelike vector}, and the events are said to be \textit{timelike separated}. Timelike separated events can be causally connected.
    \item If \( \dd{s}^2 = 0\), then \( \dd{x}^{\mu}\) is said to be a \textit{lightlike vector}, and the events are said to be \textit{lightlike separated}.
    \item If \( \dd{s}^2 < 0\), then \( \dd{x}^{\mu}\) is said to be a \textit{spacelike vector}, and the events are said to be \textit{spacelike separated}. Spacelike separated events cannot be causally connected.
\end{itemize}

We can also define a quantity called the four-momentum as
\beq p = \left(E,p^1,...,p^{D-1}\right) = \left(E,\vb{p}\right)\eeq
Taking the scalar product \(p\cdot p\) we get a quantity which we call \(-(M)^2 \), therefore
\beq[eq:p2m20constrain] p^2 = - M^2 = -E^2 + \vb{p}\cdot\vb{p}\eeq
where \(M\) can sometimes be interpreted, for example, as the mass of a particle or string.\par

\subsection{Light-cone coordinates}

In special relativity its useful to use \textit{light-cone coordinates}. The light-cone coordinates are defined
as
\beq[eq:lccoor] x^+ \doteq \frac{1}{\sqrt{2}}(x^0 + x^1) \qq{and} x^- \doteq \frac{1}{\sqrt{2}}(x^0 - x^1)\eeq
And by direct substitution, the scalar product between two vectors becomes
\beq[eq:lcscalarproduct] -\dd{s}^2 = -2\dd{x}^+\dd{x}^- + (\dd{x}^2)^2 +...+(\dd{x}^{D-1})^2\eeq
and the flat metric becomes, in the \(D=4\) case,
\beq[eq:lcflatmetric] \hat{\eta}_{\mu \nu} =
\begin{pmatrix}
0 & -1 & 0 & 0\\
-1 & 0 & 0 & 0\\
0 & 0 & 1 & 0\\
0 & 0 & 0 & 1
\end{pmatrix}\eeq\par

It's important to define \(x^+\) as the light cone time (it could also be \(x^-\), but it's not conventional) since it remains constant for a light ray in the negative \(x^1\) direction.\par

We can do the same thing to momentum and adopt momentum light-cone coordinates. Now, to find which component should be identified as the light-cone energy we must recall that energy and time are conjugated variables. If we expand the scalar product \(p\cdot x\) in both cartesian coordinates and light-cone coordinates, we find that the momentum component conjugated to \(x^+\) is \(-p_+=p^-\), thus we conclude that the light-cone energy is
\beq[] p^- = E_{lc}\eeq

\section{Conservation}

\subsection{Noether's Theorem}

The action is the integral of the Lagrangian over all it's parameters \(\{\sigma\}\), spacelike and timelike
\beq S = \int \dd^{D}\sigma \mc{L}\eeq
where \(D = d+1\) and \(d\) is the number of spacelike parameters needed to describe the system (we are considering only one timelike parameter).\par

Let the Lagrangian be a function of a field and its first order derivatives
\beq \mc{L} = \mc{L}\left(\phi^{\mu},\del_{\alpha}\phi^{\mu};\{ \sigma^{\alpha}\} \right) \qq{where} \alpha = 0,1,...,d\eeq\par

Noether's theorem says that for every continuous symmetry of the action there's a conserved current. For every conserved current there corresponds a conserved charge with respect to the timelike parameter, this charge is the integral of the current over all spacelike parameters.\par

If the variation is continuous, it admits an infinitesimal form. Therefore we can let the variation of the field be
\beq[] \var{\phi^{\alpha}}=\epsilon^ih_i^{\alpha}(\phi) \eeq
where \(\epsilon^i\) are a set of infinitesimal constants and \(i\) labels the different components the function \(h\) can have.\par

This last statement is important, since enables us to find locally conserved currents (instead of globally conserved currents).\par

The symmetry of the action with respect to the variation implies that the following quantities are conserved
\beq[eq:noethercurrent] \epsilon^ij^{\alpha}_i = \pdv{\mc{L}}{( \del_{\alpha}\phi^{\mu})} \var{\phi^{\mu}}\qq{with} \del_{\alpha}j^{\alpha}_i=0\eeq\par

The associated conserved charge is define as
\beq[eq:conschargeint] Q_i = \int \dd{\sigma^1}\dd{\sigma^2}...\dd{\sigma^d} j_i^0 \eeq
and it follows that
\beq[] \dv{Q_i}{\sigma^0}=0 \eeq

In two dimensions, \eqref{eq:conschargeint} can be generalised as the current flux across any path \(\gamma\) that goes from \(\sigma = 0\) to \(\sigma = \sigma\)
\beq[] Q_i = \uint[\gamma] \left(j^{\sigma^0}_i\dd{\sigma^1} - j^{\sigma^1}_i\dd{\sigma^0} \right) \eeq
Of course we can take a path where \(\sigma_0 \) is constant, resulting in our original definition in 2 dimensions.

\subsection{Momentum}

For a relativistic string, if the variation is given by
\beq[] \var{X^{\mu}} = \epsilon^{\mu}\eeq
where \(\epsilon^{\mu}\) is a constant, we get that the current is
\beq[] j = \left(\Pi^{\tau},\Pi^{\sigma} \right) \eeq
and the conservation of currents is just
\beq[] \pdv{\Pi^{\tau}}{\tau} + \pdv{\Pi^{\sigma}}{\sigma} = 0\eeq
which is just the equation of motion for the string.\par
We can integrate the current and get the conserved charge
\beq[]  p(\tau) = \uint[\gamma] \left( \Pi^{\tau}\dd{\sigma} - \Pi^{\sigma}\dd{\tau} \right)\eeq
which is the spacetime momentum carried by the string, since the symmetry observed corresponds to a spacetime translational invariance. If we choose a path \(\gamma\) where \(\tau\) is constant, then \(\dd{\tau} =0\) and
\beq[eq:momentumsigmaint] p(\tau) = \int_0^{\sigma_1} \Pi^{\tau}\dd{\sigma} \eeq
and we can conclude that \(\Pi^{\tau} \) is the \(\sigma\) density of spacetime momentum carried by the string.\par

Using the equations of motion \eqref{eq:PiEOM} we have that
\beq[eq:momentumconservation] \dv{p}{\tau} = -\eval{\Pi^{\sigma}}^{\sigma_1}_{0} \eeq
where \(0\) and \(\sigma_1\) are the endpoints of the string. This shows that the boundary conditions \eqref{eq:stringbc} will be of extreme importance to know if the momentum of a string is conserved.\par

% We can speculate: for closed strings \(\sigma_1\) corresponds to the same point \(0\), therefore momentum is always conserved, and that for open strings, if we impose free boundary conditions, we have that \( \Pi^{\sigma}_{\mu}=0\), we get that the momentum is conserved with respect to \(\tau\)
% \beq[] \dv{p_{\mu}}{\tau}=0\eeq
% It's important to note that if we impose Dirichlet boundary conditions, fixed endpoints, spacetime momentum is not conserved. This result will give rise to the \(D\)-branes, which will be studied later.\par

\subsection{Lorentz current and charge}

Now it's time to look at the conserved currents and charges of a very special continuous transformation, the Lorentz transformation. But first we must find the infinitesimal form of a general Lorentz transformation.\par

Since the Lorentz transformations are linear transformations, we can start using the fact that every infinitesimal linear transformation is of the form
\beq[eq:lorentzvar] \var{a^{\mu}} = \epsilon^{\mu \nu} a_{\nu} \qq{and also} \var{b^{\mu}} = \epsilon^{\mu\nu} b_ {\nu} \eeq
where \( \epsilon^{\mu \nu}\) is a matrix of infinitesimal constants, analog to the previous \(\epsilon^i\). Invariance of the scalar product upon such variation imposes that
\beq[] \var{\left( \eta_{\mu \nu}a^{\mu}b^{\nu} \right)} = 0 \eeq
Developing the above expression we find
\beq[] \eta_{\mu\nu}\left(\var{a^{\mu}}\right)b^{\nu} + \eta_{\mu\nu}a^{\mu}\left(\var{b^{\nu}}\right)=0 \eeq
\beq[] \eta_{\mu\nu}\left(\epsilon^{\mu\nu}a_ {\nu}\right)b^{\nu} + \eta_{\mu\nu}a^{\mu}\left(\epsilon^{\nu\mu}b_{\mu}\right)=0  \eeq
and since \(a_{\nu}b^{\nu} = a^{\mu}b_{\mu} = a\cdot b\) we can group the similar terms and get
\beq[] \eta_{\mu\nu}(a\cdot b) \left( \epsilon^{\mu\nu}+\epsilon^{\nu\mu}\right)=0 \eeq
and since the above expression must be true for any \(a\) and \(b\), we find that the condition we must impose on \(\epsilon^{\mu\nu}\) is that
\beq[]  \epsilon^{\mu\nu}+\epsilon^{\nu\mu} = 0\eeq
which means that for an infinitesimal linear transformation \(\epsilon^{\mu\nu}\) to be an infinitesimal Lorentz transformation, \(\epsilon^{\mu\nu}\) must be an \textit{antisymmetric} matrix.\par

Now, using \eqref{eq:noethercurrent} with \eqref{eq:lorentzvar} and the antisymmetry of \(\epsilon^{\mu\nu}\) we find that the associated conserved currents are (choosing a normalization)
\beq[] \mc{M}^{\alpha}_{\mu\nu} = X_{\mu}\Pi^{\alpha}_{\nu} - X_{\nu}\Pi^{\alpha}_{\mu} \qc \alpha = \tau,\sigma \eeq
which is also antisymmetric. The conservation of current is then
\beq[] \pdv{\mc{M}^{\tau}_{\mu\nu}}{\tau} + \pdv{\mc{M}^{\sigma}_{\mu\nu}}{\sigma} =0 \eeq
and the charges matrix are
\beq[] M_{\mu\nu} = \uint[\gamma] \left( \mc{M}^{\tau}_{\mu\nu}\dd{\sigma} - \mc{M}^{\sigma}_{\mu\nu}\dd{\tau} \right) \eeq\par

Since we can choose any path \(\gamma\) to compute the charges, we'll choose a constant \(\tau\) path, leaving the Lorentz charges matrix as
\beq[eq:lorentzchargematrix] M_{\mu\nu} = \int_0^{\sigma_1} \mc{M}^{\tau}_{\mu\nu} \dd{\sigma} = \int_0^{\sigma_1} \left(X_{\mu}\Pi^{\tau}_{\nu} - X_{\nu}\Pi^{\tau}_{\mu} \right)\dd{\sigma} \eeq
Being antisymmetric, in \(D\) spacetime dimensions we have \(\hlf D(D-1)\) different Lorentz charges.\par
In four spacetime dimensions we have: three charges associated with the three basic boosts, and three charges associated with the three basic rotations. Letting \(i\) and \(j\) be spatial indices, to find the charges we must compute \( M^{0i}\) and \(M^{ij}\).\par

Since \(\Pi^{\tau}_i\) is the momentum density, \(\mc{M}^{\tau}_{ij}\) fits the angular momentum density definition, so the charges associated relate to the string angular momentum as1
\beq[] L_i = \hlf \varepsilon^{ijk}M_{jk} \eeq\par

The charges \(M^{0i}\) associated with the boosts are
\beq[] M^{0i} = \int_0^{\sigma_1} \left( t\Pi^{\tau i} - X^i\Pi^{\tau 0} \right)\dd{\sigma} = ctp^i - \int_0^{\sigma_1}X^i\Pi^{\tau 0}\dd{\sigma}  \eeq\par

The importance of these Lorentz charges cannot be overstated: upon quantisation some symmetries may be lost, but we cannot let this happen to Lorentz symmetry since we want our string theory to be relativistic! This Lorentz charges will be responsible for imposing some serious properties on the quantum relativistic string. We'll see them later.

\section{Choosing a gauge}

The \textit{gauge} is the way a set of parameters \(\{\sigma\}\) used to the describe a physical system relates to it's values, which in our case is the way \(\tau\) and \(\sigma\) relates to the string coordinates \(X^{\mu}\). Since our Nambu-Goto action has diff-invariance (\eqref{eq:diffinvar1} and \eqref{eq:diffinvar2}), we can choose our gauge freely, and this is good since some calculations are a lot easier when held with the appropriate gauge, as we'll see.\par

Let \(n^{\mu}\) be a vector. Our starting point is: we want the amount of \(n\cdot p\) momentum the string carries from \(\sigma=0\) to \(\sigma\) to be directly proportional to \(\sigma\), and this is accomplished by imposing the constancy of the \(n\cdot p\) density, \(n\cdot \Pi^{\tau}\), so that, according to \eqref{eq:momentumsigmaint}
\beq[eq:npsigma] n\cdot p (\sigma) = \int_0^{\sigma}n\cdot \Pi^{\tau}\dd{\sigma}=n\cdot \Pi^{\tau}*\sigma \eeq\par

We also want the vector \(n^{\mu}\) in a way that \(n\cdot p\) is \textit{\textbf{always}} conserved with respect to \(\tau\)
\beq[] \dv{\left(n\cdot p\right)}{\tau} =0 \eeq
According to \eqref{eq:momentumconservation}, to accomplish such conservation we must impose the condition
\beq[] n\cdot \Pi^{\sigma} =0 \qq{at the string endpoints} \eeq
but if we dot the equations of motion \eqref{eq:PiEOM} with \(n\), we get
\beq[] \pdv{\tau}\left(n\cdot \Pi^{\tau}\right) + \pdv{\sigma}\left(n\cdot\Pi^{\sigma}\right)  = 0\eeq
\beq[] \pdv{\tau}\left(\frac{n\cdot p }{\sigma_1}\right) + \pdv{\sigma}\left(n\cdot\Pi^{\sigma}\right)  = 0 \eeq
\beq[] \pdv{\sigma} \left(n\cdot \Pi^{\sigma} \right) = 0\eeq
so we find that \( n\cdot \Pi^{\sigma}\) does not depend on \(\sigma\)! Therefore if we find
\beq[eq:nsigmacondition] n\cdot \Pi^{\sigma}=0 \qq{at \textit{\textbf{any}} point on the string} \eeq
the conservation of \(n\cdot p\) is ensured.\par

Considering every situation in which there are solutions to \eqref{eq:nsigmacondition}, we want to set the following gauge on \(\sigma\)
\beq[eq:sigmagauge] n\cdot \Pi^{\tau} = \frac{n\cdot p}{\sigma_1} \eeq
and the following gauge on \(\tau\)
\beq[eq:taugauge] n\cdot X = \frac{n\cdot p}{T\sigma_1} \tau \eeq\par

Now, lets search for the solutions of \eqref{eq:nsigmacondition}. Writing \(\Pi^{\sigma}\) explicitly we have
\beq[] n\cdot \Pi^{\sigma} = -T \frac{\left(\dot{X}\cdot X'\right) \del_{\tau}(n\cdot X) - \left(\dot{X}\right)^2 \del_{\sigma}(n\cdot X) }{\sqrt{(\dot{X}\cdot X')^2 - (\dot{X})^2(X')^2}} \eeq
Since \(n\cdot X\) is only a function of \(\tau\), it follows that \(\del_{\sigma}\left(n\cdot X \right) = 0\), therefore
\beq[] n\cdot \Pi^{\sigma} = -T \frac{\left(\dot{X}\cdot X'\right) \del_{\tau}(n\cdot X) }{\sqrt{(\dot{X}\cdot X')^2 - (\dot{X})^2(X')^2}} \eeq
So in order to satisfy \eqref{eq:nsigmacondition} we must find
\beq[eq:gaugeXXconstrain] \dot{X}\cdot X' = 0 \qq{at some point on the string} \eeq\par

Using \eqref{eq:gaugeXXconstrain} on \(n\cdot \Pi^{\tau}\) we have, explicitly
\beq[] n\cdot \Pi^{\tau} = -T \frac{\left(\dot{X}\cdot X'\right) \del_{\sigma}(n\cdot X) - \left(X'\right)^2 \del_{\tau}(n\cdot X) }{\sqrt{(\dot{X}\cdot X')^2 - (\dot{X})^2(X')^2}} \eeq
\beq[] n\cdot \Pi^{\tau} = T \frac{\left(X'\right)^2 \del_{\tau}(n\cdot X) }{\sqrt{ - (\dot{X})^2(X')^2}} \eeq
and substituting \eqref{eq:taugauge} and \eqref{eq:sigmagauge}
\beq[] \frac{n\cdot p}{\sigma_1} = T \frac{X'^2 }{\sqrt{ - (\dot{X})^2(X')^2}} \pdv{\tau}\left( \frac{n\cdot p}{T\sigma_1}\tau\right) \eeq
leaving us with only
\beq[] 1 = \frac{X'^2}{\sqrt{ - (\dot{X})^2(X')^2}} \eeq
which is just
\beq[eq:gaugeX2constrain] \dot{X}^2 + X'^2 = 0 \eeq\par

This two constrains \eqref{eq:gaugeXXconstrain} and \eqref{eq:gaugeX2constrain} can be summarised as
\beq[eq:gaugeconstrain] \left(\dot{X} \pm X' \right)^2 = 0 \eeq\par

With the use of \eqref{eq:gaugeXXconstrain} and \eqref{eq:gaugeX2constrain}, the expressions for \(\Pi^{\sigma}\) and \(\Pi^{\tau}\) simplify tremendously:
\beq[] \Pi^{\tau} = T \dot{X} \eeq
and
\beq[] \Pi^{\sigma} = -TX' \eeq
and the equations of motion \eqref{eq:PiEOM} become
\beq[eq:waveequation] \ddot{X} - X" = 0 \eeq
which is just the well-known \textit{wave equation}!\par

It's important to understand what this all means in practice: you simply use the gauge \eqref{eq:sigmagauge} and \eqref{eq:taugauge} to get the equations of motion to be the wave equation \eqref{eq:waveequation}, and then make sure the solutions you find satisfy \eqref{eq:gaugeXXconstrain} and \eqref{eq:gaugeX2constrain}. Of course, if the solution you find doesn't satisfy \eqref{eq:gaugeXXconstrain} and \eqref{eq:gaugeX2constrain}, then your solution is no good.\par

\subsection{Some notable results using our gauge}

Using our gauge constrains \eqref{eq:gaugeXXconstrain} and \eqref{eq:gaugeX2constrain} on some previous results we found it becomes a lot easier to interpret them. Lets take our momentum \eqref{eq:momentumsigmaint}. It becomes simply
\beq[eq:pgauge] p(\tau) = T\int_0^{\sigma_1}\dot{X}\dd{\sigma} \eeq
and its conservation
\beq[eq:pconsrvgauge] \dv{p}{\tau} = -T \eval{X'}^{\sigma=\sigma_1}_{\sigma=0}\eeq
and together with the boundary term, which in our gauge becomes
\beq[eq:bcgauge] \int_{\tau_f}^{\tau_i}\dd{\tau} \eval{\left[X'_{\mu}\var{X}^{\mu}\right]}^{\sigma=\sigma_1}_{\sigma = 0} =0  \eeq
we are able to better relate momentum conservation with some of the acceptable boundary conditions which will be seen on the next chapter.\par

% \subsection{Slope parameter}

% Now, in order to perform the following calculations we will set
% \beq[] \tau = t \eeq
% known as the \textit{static gauge}. Note that in this gauge the \(X^0\) component is already solved.\par

% Consider a rigidly rotating open string. The motion of such string is comprised in two dimensions, so we can save up time writing only its \(X^1\) and \(X^2\) components. Solving the wave equation \eqref{eq:waveequation} our solution is
% \beq[eq:rotatingstring] \left(X^1,X^2 \right) = \frac{\sigma_1}{\pi}\cos{\frac{\pi \sigma}{\sigma_1}}\left(\cos{\frac{\pi t}{\sigma_1}} , \sin{\frac{\pi t}{\sigma_1}}\right) \eeq\par

% Plugging \eqref{eq:rotatingstring} into \eqref{eq:lorentzchargematrix} we find that the only non-vanishing component of \(M_{\mu\nu}\) is \(M_{12}\), which when computed gives
% \beq[eq:rotatingangularmomentum] M_{12} = \frac{\sigma_1^2T}{2\pi} \eeq\par

% The difference between a rigidly rotating string and a stretched string is only that the former moves and the latter doesn't, so we can expect its potential energy to be the same! To simplify our calculations, lets consider now just a stretched string. In this case, using the static gauge, the only non-trivial component to be found is \(X^1\), but we do know something about it. Since the string doesn't move, \(X^1\) can only be a function of \(\sigma\), lets call it \(f(\sigma)\). We also know the boundaries of \(X^1\), which in terms of \(f\) read
% \beq[] f(0) = 0 \qq{and} f(\sigma_1) = a \eeq
% where \(a\) is where one string endpoint lies. Since the stretched string is comprised in a single dimension, it's best to just say that the string lies in one of the \(d\) spatial spacetime axes (we already did that when we said that only \(X^1\) mattered). With this interpretation \(a\) is just the length of the string while \(f\) tells the \textit{speed} of the \(\sigma\) parameterization.\par

% Lets also see what happens to our Nambu-Goto action \eqref{eq:nambugotoaction} when we plug in \eqref{eq:gaugeXXconstrain} and \eqref{eq:gaugeX2constrain}. We get:
% \beq[eq:nambugotoactiongauge] S[X] = -T \int_{\tau_i}^{\tau_f}\dd{\tau}\int_0^{\sigma_1}\dd{\sigma} \left(\pdv{X}{\sigma}\right)^2 \eeq
% and computing \(\dot{X}\) and \(X'\) and plugging them in \eqref{eq:nambugotoactiongauge} we find that the action actually is
% \beq[] S[X] = -T \int_{t_i}^{t_f}\dd{t}\int_0^{\sigma_1}\dd{\sigma}\dv{f}{\sigma} \eeq
% \beq[eq:stretchedaction] S[X] = \int_{t_i}^{t_f} \dd{t} \left(-Ta \right) \eeq
% Setting \(a=\sigma_1\) for simplicity, we can now identify the integrand of \eqref{eq:stretchedaction} with the string Lagrangian, recalling that since the string is not moving there's no kinetic energy (\(K=0\))
% \beq[] \left(-T\sigma_1 \right) = L = K - V = - V\eeq
% so now we know what's the string rest energy, the energy due only to its \textit{own existence, regardless of its motion}
% \beq[eq:stringrestenergy] E = K + V = V = T\sigma_1 \eeq\par

% To end this section: lets plug \eqref{eq:stringrestenergy} into \eqref{eq:rotatingangularmomentum} to get
% \beq[] J = M_{12} = \frac{1}{2\pi T}E^2 = \alpha' E^2 \eeq
% where \(\alpha'\) is called the \textit{slope parameter}. The slope parameter \(\alpha'\) is preferred over the string tension \(T\), and we'll be using it a lot.\par

\nocite{*}
\end{document}
