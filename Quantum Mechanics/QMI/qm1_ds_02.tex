\documentclass{_mypackages/monograph}

\title{Quantum Mechanics I \\ Discussion Set 02} % \MyTitle
\author{Bruno Murino - 8944901} % \MyAuthor
\date{\today} % \MyDate

\addbibresource{qinfo.bib}
\graphicspath{ {figures/} }

\begin{document}
% \frontmatter

\solutionstp
% \dominitoc
% \doparttoc
% \pagestyle{onlypagenum}
% \tableofcontents
% \mainmatter

\chapter*{2.}

Let
\begin{equation}
    \eho = \frac{1}{4}\mqty(2 & 1 & 1 \\ 1 & 1 & 0 \\ 1 & 0 & 1).
\end{equation}

\section*{(a)}




\chapter*{3.}


\chapter*{4.}
\section*{(a)}
Lets denote
\begin{equation}
    \ket{+} = \ket{0} = \mqty(1 \\ 0) \qand \ket{-} = \ket{1} = \mqty(0 \\ 1)
\end{equation}

Let
\begin{equation}
    \rho_1 = \hlf (\ketbra{11} + \ketbra{00})
\end{equation}
and
\begin{equation}
    \rho_2 = \hlf (\ket{11} + \ket{00})(\bra{11} + \bra{00}).
\end{equation}

To see if \(\rho_1\) represents a pure state we can compute its purity \(\mathcal{P}_1 = \Tr(\rho_1^2)\). We start by computing \(\rho_1^2\):
\begin{equation}
\begin{split}
    \rho_1^2 &= \frac{1}{4}(\ketbra{11} + \ketbra{00})(\ketbra{11} + \ketbra{00}) \\
    &= \frac{1}{4}\bigg[\ket{11}\braket{11}\bra{11} + \ket{11}\braket{11}{00}\bra{00} + \ket{00}\braket{00}{11}\bra{11} + \ket{00}\braket{00}\bra{00}\bigg],
\end{split}
\end{equation}
and since 
\begin{equation}
    \braket{ii}{jj} = \delta_{ij},
\end{equation}
we find that
\begin{equation}
    \rho_1^2 = \frac{1}{4}\bigg[\ketbra{11} + \ketbra{00}\bigg] = \frac{1}{4} \mqty(\dmat[0]{1,0,0,1})
\end{equation}
meaning that
\begin{equation}
    \mathcal{P}_1 = \hlf \neq 1,
\end{equation}
which implies that \(\rho_1\) does not represent a pure state.

For \(\rho_2\), on the other hand, its quite trivial to see that if
\begin{equation}
    \ket{\psi} = \frac{\ket{11} + \ket{00}}{\sqrt{2}},
\end{equation}
then
\begin{equation}
    \rho_2 = \ketbra{\psi},
\end{equation}
which is the precise definition of a pure state. Nonetheless, we can compute its purity too. Computing \(\rho_2^2\) we already find that
\begin{equation}
    \rho_2^2 = \ket{\psi}\braket{\psi}\bra{\psi} = \ketbra{\psi} = \rho_2,
\end{equation}
and since \(\Tr(\rho_2)=1\), because \(\rho_2\) \emph{is} a density matrix, we already know that
\begin{equation}
    \mathcal{P}_2 = \Tr(\rho_2^2) = \Tr(\rho_2) = 1,
\end{equation}
meaning that \(\rho_2\) represent a pure state indeed.

\section*{(b)}

To compute what \(B\) sees we need to find the reduced density matrix \(\rho^B\), which is the partial trace of \(\rho\) over \(B\). Lets compute \(\rho^B\) for both situations. Using the fact that
\begin{equation}
    \Tr_B(\ketbra{a,b}{a',b'}) = \ketbra{a}{a'}\delta_{bb'},
\end{equation}
we find that
\begin{equation}
\begin{split}
    \rho_1^B = \frac{1}{2}(\ketbra{1} + \ketbra{0}) = \frac{1}{2}\mqty(\dmat[0]{1,1}), \\
    \rho_2^B = \frac{1}{2}(\ketbra{1} + \ketbra{0}) = \frac{1}{2}\mqty(\dmat[0]{1,1}).
\end{split}
\end{equation}
It's important to notice that they are equal, which means that the observer \(B\) won't be able to tell the difference between the situations! Anyway, lets compute \(\ev{S_x}_B\). Lets denote \(\rho_1^B = \rho_2^B = \rho^B\). Recalling that
\begin{equation}
    \ev{S_x}_B = \Tr(S_x \rho^B),
\end{equation}
we need to compute \(S_x \rho^B\). Since
\begin{equation}
    S_x =  \frac{1}{2}\mqty(\admat[0]{1,1}),
\end{equation}
we find that
\begin{equation}
    S_x \rho^B = \frac{1}{4}\mqty(\admat[0]{1,1}),
\end{equation}
and then
\begin{equation}
    \ev{S_x}_B = \Tr(S_x \rho^B) = 0,
\end{equation}
meaning that \(B\) always measures \(0\).

\chapter*{5. WHAT}

The relation
\begin{equation}
    \Delta A \Delta B \geq \hlf \abs{\ev{\comm{A}{B}}}
\end{equation}
is the so called \emph{uncertainty relation}.

Given a state \(\ket{\phi}\) such that
\begin{equation}
    (A - \ev{A})\ket{\phi} = k (B-\ev{B})\ket{\phi},
\end{equation}
we can rearrange the terms and find that
\begin{equation}\label{eq:1}
    (A-kB)\ket{\phi} = (\ev{A} - k\ev{B})\ket{\phi}.
\end{equation}
Applying \(\bra{\phi}A\) from the left on \eqref{eq:1} we find that
\begin{equation}\label{eq:evAB}
    \ev{AB}{\phi} = \ev{A}\ev{B} + \frac{(\Delta A)^2}{k}
\end{equation}
and applying \(\bra{\phi}B\) from the left of \eqref{eq:1} we find that
\begin{equation}\label{eq:evBA}
    \ev{BA}{\phi} = \ev{A}\ev{B} + k(\Delta B)^2.
\end{equation}
If we do \eqref{eq:evAB} - \eqref{eq:evBA}, we find that
\begin{equation}
    \ev{\comm{A}{B}} = \frac{(\Delta A)^2}{k} - k (\Delta B)^2
\end{equation}






























% \backmatter
% \printbib
\end{document}