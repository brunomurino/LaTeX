\documentclass[oneside, 12pt]{book}

\usepackage{mypreamble}
\usepackage[backend=biber,style=nature]{biblatex}
%\cite{barton} = [1] and \footfullcite[p. 324]{barton} = ^1 + footnote, p. 324 (e.g.)

\usepackage{mycommands}
\usepackage{mytheme1}

\addbibresource{ref_EM.bib}

%--------------------------------------------------------------



%--------------------------------------------------------------
\begin{document}
\pagestyle{mypage2}
\chapter{Potentials and Fields} 

Electrostatics is a situation defined by constant electric fields (stationary charges), so there are no magnetic fields and Faraday's law becomes
\beq[] \curl{\vb{E}} = 0 \eeq
which is always satisfied by an electric field \(\vb{E}\) such that
\beq[eq:epotfield] \vb{E} = -\gradi{\epot} \eeq
since
\beq[] \curl{\left(\grad{\epot} \right)} \equiv 0 \eeq
In terms of the electric potential \(\epot\) Gauss's law becomes
\beq[eq:laplacianepot] \laplacian{\epot} = -\frac{\rho}{\epsilon_0} \eeq

\section{Magnetostatics}

\end{document}