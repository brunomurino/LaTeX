\documentclass[oneside, 12pt]{book}

\usepackage{mypreamble}
\usepackage{mycommands}
\usepackage{mytheme1}

\begin{document}

\chapter{The Dirac Delta Function} \edef\TheDiracDeltaFunctionChapter{\thechapter}

\beq[eq:divdelta] \div{\left(\frac{\vb{x}-\vb{x'}}{\abs{\vb{x}-\vb{x'}}^3}\right)} = 4\pi \delta(\vb{x}-\vb{x'})\eeq
And as a consequence of 
\beq[eq:grad1r] \gradi{\left( \frac{1}{r}\right)} = -\frac{\vb{r}}{r^3} = -\frac{\hat{r}}{r^2}\eeq
we have
\beq[eq:lapdelta] \laplacian\left( \frac{1}{\abs{\vb{x}-\vb{x'}}}\right) = -4\pi \delta(\vb{x}-\vb{x'}) \eeq
The following integrals are only satisfied if the point in which the argument of the delta function becomes zero is included in the region of integration, otherwise, the integral vanishes.
\beq[eq:intedelta] \int \dd{x} \delta(x) = 1\eeq

\beq[eq:deltapick] \int \dd{x}f(x)\delta(x - x') = f(x')\eeq

\beq[] \delta(ax) = \frac{1}{\abs{a}}\delta(x) \eeq

\end{document}