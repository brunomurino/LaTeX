\documentclass[oneside, 12pt]{article}

\usepackage{mypreamble}
% \usepackage[backend=biber,style=nature]{biblatex}
%\cite{barton} = [1] and \footfullcite[p. 324]{barton} = ^1 + footnote, p. 324 (e.g.)

\usepackage{mycommands}
\usepackage{mytheme1}

% \addbibresource{ref_ST.bib}

%--------------------------------------------------------------

\linespread{1} 
\geometry{a4paper, total={170mm,257mm}, left=20mm, top=20mm,}

%--------------------------------------------------------------

\begin{document}

\import{String Theory/}{ST_capa2.tex}
 
\pagestyle{mypage2} \normalfont

\section{Resumo}

O objetivo desta iniciação científica é expor uma introdução básica à Teoria das Cordas. Maestria completa de um assunto tão vasto requereria, naturalmente, um curso de pós-graduação em Física (e Matemática, em certos aspectos) e muitos anos de estudo, mas algumas de suas noções básicas podem ser entendidas com o que se é feito nos primeiros dois ou três anos de graduação. Além disso, a tarefa de introduzir a Teoria das Cordas se torna muito mais viável graças ao excelente e bem-testado livro de graduação “A first course in String Theory” por Barton Zwiebach do Massachussets Institute of Technology, livro este que segui à risca. Nota que este relatório é parcial referente ao período de 1 de Abril de 2016 à 30 de Abril de 2017. \par 

\section{Continuação da Quarta fase}

No capítulo 22 foi estudada a mecânica estatística da corda. É feito um breve resumo geral da mecânica estatística e então é encontrada a temperatura de Hagedorn. É derivada também a função de partição da corda. Mais um trunfo da teoria das cordas se manifesta quando ela possibilita o estudo da entropia de buracos negros. Foi feito apenas o primeiro exercício deste capítulo.\par 

No capítulo 23 é descrito brevemente a correspondência AdS/CFT, onde a teoria das cordas supersimétrica é dita equivalente à teoria supersimétrica de Yang-Mills em 4 dimensões. Nenhum exercício deste capítulo foi feito.\par 

\section*{Referências}

\begin{enumerate}
    \item \textit{A First Course in String Theory}, Barton Zwiebach
\end{enumerate}

% \section{Outras atividades}

% Durante o período da iniciação científica, foi feito uma espécie de resumo do conteúdo aprendido, contemplando aproximadamente os capítulos 1 ao 13 e 15. Este resumo

%\backmatter

% \nocite{*}
% \printbibliography[title={\texorpdfstring{\textsc{\textbf{Referências}}}{Referências}}] 
% \addcontentsline{toc}{chapter}{\textsc{References}} 

\end{document}