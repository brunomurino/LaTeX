\documentclass{_mypackages/monograph}

\title{Quantum Mechanics I \\ Problem Set 06} % \MyTitle
\author{Bruno Murino - 8944901} % \MyAuthor
\date{\today} % \MyDate

\addbibresource{qinfo.bib}
\graphicspath{ {figures/} }

\begin{document}
% \frontmatter

\solutionstp
% \dominitoc
% \doparttoc
% \pagestyle{onlypagenum}
% \tableofcontents
% \mainmatter

% \import{../}{usefulequations}

\chapter*{1.}

We know that
\begin{equation}\label{eq:cgcoef}
    \ket{j_1,j_2,J,M} = \sum_{m_1,m_2} \braket{j_1,j_2,m_1,m_2}{j_1,j_2,J,M} \ket{j_1,j_2,m_1,m_2},
\end{equation}
where \(\braket{j_1,j_2,m_1,m_2}{j_1,j_2,J,M}\) are the Clebsch-Gordan coefficients. The non-vanishing coefficients obey the following conditions
\begin{equation}\label{eq:addofangmom}
    \left\{\begin{aligned}
        m_1 +  & \, m_2 = M,  \\
        \abs{j_1-j_2} \leq \, &J \leq j_1 + j_2, \\
        j_1+j_2 + &\, J = \text{integer}.
       \end{aligned}
    \right.
\end{equation}
 Let's omit \(j_1,j_2\) from the notation since they are fixed.
 
\subsubsection{First method}

We want to find the Clebsch-Gordan coefficients for \(j_1=j_2=1\). We know that \(m_1,m_2=0,\pm1\). Also, for \(j_1=j_2=1\) we can have \(J=0,1,2\). Let's start with \(J=2\). Then, the allowed \(M\) are \(M=0,\pm1,\pm2\). If we choose \(M=2\), then the allowed \(m_1\) and \(m_2\) are \(m_1 = m_2 = 1\). Since these are the only possible values for \(m_1,m_2\), the sum in \eqref{eq:cgcoef} has only one term:
\begin{equation}
    \ket{2,2} = \braket{1,1}{2,2}\ket{1,1},
\end{equation}
and due to normalisation we require \(\abs{\braket{1,1}{2,2}}=1\) and then choose \(\braket{1,1}{2,2}\) to be positive, meaning that \(\braket{1,1}{2,2}=1\):
\begin{mybox}
\begin{equation}
    \ket{2,2} = \ket{1}\tens\ket{1} = \ket{1,1}.
\end{equation}
\end{mybox}
Applying \(J_- = J_- \tens \idm + \idm \tens J_-\) we obtain
\begin{equation}
    2\ket{2,1} = \sqrt{2}\ket{0,1} + \sqrt{2}\ket{1,0}
\end{equation}
and then
\begin{mybox}
\begin{equation}
    \ket{2,1} = \frac{1}{\sqrt{2}}(\ket{0,1}+\ket{1,0}).
\end{equation}
\end{mybox}

Applying \(J_- = J_- \tens \idm + \idm \tens J_-\) again we obtain
\begin{equation}
\begin{split}
    a_-(2,1)\ket{2,0} &=\\
    \frac{1}{\sqrt{2}}\Bigg[a_-(1,0)\ket{-1,1} &+ \Big(a_-(1,1)+a_-(1,1)\Big)\ket{0,0} + a_-(1,0)\ket{1,-1}\Bigg],
\end{split}
\end{equation}
and then
\begin{mybox}
\begin{equation}
    \ket{2,0} = \frac{1}{\sqrt{6}}\Bigg[\ket{-1,1} + 2\ket{0,0} + \ket{1,-1}\Bigg].
\end{equation}
\end{mybox}

Again applying \(J_- = J_- \tens \idm + \idm \tens J_-\) we obtain
\begin{equation}
\begin{split}
    a_-(2,0)\ket{2,-1}= \\
    \frac{1}{\sqrt{6}}\Bigg[a_-(1,1)\ket{-1,0} &+ 2a_-(1,0)\Big(\ket{-1,0} + \ket{0,-1} \Big)      + a_-(1,1)\ket{0,-1}\Bigg],
\end{split}
\end{equation}
and then
\begin{mybox}
\begin{equation}
    \ket{2,-1} = \frac{1}{\sqrt{2}}\Bigg[\ket{-1,0} + \ket{0,-1} \Bigg].
\end{equation}
\end{mybox}

Using the same argument we did to obtain \(\ket{2,2}\) we can obtain \(\ket{2,-2}\). The only combination of \(m_1,m_2\) that gives \(M=-2\) is \(m_1=m_2=-1\), thus \eqref{eq:cgcoef} only has one term, which is \(\pm1\) due to normalisation and we choose \(+1\) to keep all the CG coefficients positive:
\begin{mybox}
\begin{equation}
    \ket{2,-2} = \ket{-1,-1}.
\end{equation}
\end{mybox}

Now, we can obtain \(\ket{1,1}\) by requiring it to be orthogonal to \(\ket{2,1}\). First, if \(M=1\) then the allowed \(m_1,m_2\) are \((m_1=0,m_2=1)\) and \((m_1=1,m_2=0)\), meaning that
\begin{equation}
    \ket{1,1} = \braket{0,1}{1,1}\ket{0,1} + \braket{1,0}{1,1}\ket{1,0}.
\end{equation}
Calling \(\braket{0,1}{1,1}=a\) and \(\braket{1,0}{1,1}=b\) and acting with \(\bra{2,1}\) on the left we obtain
\begin{equation}
    \braket{2,1}{1,1} = \frac{1}{\sqrt{2}}(a\braket{01} + b \braket{1,0}) = 0
\end{equation}
from which we get that \(a=-b\). Also, due to the normalisation of \(\ket{1,1}\) we need
\begin{equation}
    \abs{a}^2 + \abs{b}^2 = 1
\end{equation}
which finally gives us \(a = -b = 1/\sqrt{2}\):
\begin{mybox}
\begin{equation}
    \ket{1,1} = \frac{1}{\sqrt{2}}\Bigg[\ket{0,1} - \ket{1,0} \Bigg].
\end{equation}
\end{mybox}

Acting with \(J_- = J_- \tens \idm + \idm \tens J_-\) we obtain
\begin{equation}
    \sqrt{2}\ket{1,0} = \frac{1}{\sqrt{2}}\Bigg[\sqrt{2}\ket{-1,1} +(\sqrt{2}-\sqrt{2})\ket{0,0}-\sqrt{2}\ket{1,-1} \Bigg],
\end{equation}
and then
\begin{mybox}
\begin{equation}
    \ket{1,0}=\frac{1}{\sqrt{2}} \Bigg[\ket{-1,1}-\ket{1,-1} \Bigg].
\end{equation}
\end{mybox}

Acting again with \(J_- = J_- \tens \idm + \idm \tens J_-\) we obtain
\begin{equation}
    \sqrt{2}\ket{1,-1} = \frac{1}{\sqrt{2}}\Bigg[ \sqrt{2}\ket{-1,0} - \sqrt{2}\ket{0,-1} \Bigg],
\end{equation}
and then
\begin{mybox}
\begin{equation}
    \ket{1,-1} = \frac{1}{\sqrt{2}}\Bigg[ \ket{-1,0} - \ket{0,-1}\Bigg].
\end{equation}
\end{mybox}

To obtain \(\ket{0,0}\) we must consider its orthogonality with \(\ket{1,0}\) and \(\ket{2,0}\). First, with \(M=0\) we can have \((m_1=1,m_2=-1)\),\((m_1=0,m_2=0)\) and \((m_1=-1,m_2=1)\), meaning that the sum in \eqref{eq:cgcoef} will have three terms, which we'll call \(a\),\(b\) and \(c\):
\begin{equation}
    \ket{0,0} = a \ket{1,-1} + b\ket{0,0} + c\ket{-1,1}.
\end{equation}
The orthogonality with \(\ket{2,0}\) gives
\begin{equation}
    0 = a + 2b + c,
\end{equation}
while the orthogonality with \(\ket{1,0}\) gives
\begin{equation}
    0 = a-c.
\end{equation}
From the normalisation of \(\ket{0,0}\) we also have
\begin{equation}
    \abs{a}^2 + \abs{b}^2 + \abs{c}^2 = 1.
\end{equation}
Using \(a=c\) we find
\begin{equation}
    b = -a
\end{equation}
and then
\begin{equation}
    3\abs{a}^2 = 1,
\end{equation}
from which we finally obtain
\begin{mybox}
\begin{equation}
    \ket{0,0} = \frac{1}{\sqrt{3}}\Bigg[ \ket{1,-1} -\ket{0,0} + \ket{-1,1}\Bigg].
\end{equation}
\end{mybox}

\subsubsection{Second method}

We start with the recursion relation
\begin{multline}
    a_\mp(J,M) \braket{m_1,m_2}{J,M\mp1} =\\
    a_\pm(j_1,m_1)\braket{m_1\pm1,m_2}{J,M} + a_\pm(j_2,m_2)\braket{m_1,m_2\pm1}{J,M}
\end{multline}
Recall that \(m_2\) must be such that \(m_1+m_2 = M\mp 1\):
\begin{multline}
    a_\mp(J,M) \braket{m_1,M\mp 1-m_1}{J,M\mp1} =\\
    a_\pm(j_1,m_1)\braket{m_1\pm1,M\mp 1-m_1}{J,M} +\\
    a_\pm(j_2,M\mp 1-m_1)\braket{m_1,M-m_1}{J,M}
\end{multline}

We have \(j_1=j_2=1\).
Let's start with \(J=1\), \(M=1\) and \(m_1=1\), and take the lower relation
\begin{equation}
    a_-(1,1)\braket{0,1}{1,1} = -a_-(1,1)\braket{1,0}{11}.
\end{equation}
For \(M=1\) we can have \((m_1=0,m_2=1)\) and \((m_1=1,m_2=0)\), then, calling
\begin{equation}
    \braket{0,1}{1,1} = \alpha 
\end{equation}
we have \(\braket{1,0}{11} = -\alpha\) and due to normalisation (and sign conventions)
\begin{equation}
    \alpha = \frac{1}{\sqrt{2}},
\end{equation}
then
\begin{mybox}
\begin{equation}
    \ket{1,1} = \frac{1}{\sqrt{2}}\Bigg[\ket{0,1} - \ket{1,0} \Bigg].
\end{equation}
\end{mybox}

Now let's obtain the expansion of \(\ket{1,0}\). For \(M=0\) there are three coefficients: the one with \(m_1=1\), one with \(m_1=0\) and one with \(m_1=-1\) (with \(m_2\) according to which relation is being considered). Taking the upper relation with \(m_1=1\) we obtain
\begin{equation}
    a_-(1,1)\braket{1,-1}{1,0} = a_+(1,-1)\braket{1,0}{1,1} = -\frac{a_+(1,-1)}{\sqrt{2}},
\end{equation}
then
\begin{equation}
    \braket{1,-1}{1,0} = -\frac{1}{\sqrt{2}}.
\end{equation}
Considering \(m_1=0\) we obtain from the upper relation
\begin{equation}
    a_-(1,1)\braket{0,0}{1,0} = a_+(1,0)\braket{1,0}{1,1} + a_+(1,0)\braket{0,1}{1,1} = a_+(1,0)\Big(\frac{1}{\sqrt{2}} - \frac{1}{\sqrt{2}} \Big) = 0,
\end{equation}
then
\begin{equation}
    \braket{0,0}{1,0} = 0.
\end{equation}
Considering \(m_1=-1\) we obtain from the upper relation
\begin{equation}
    a_-(1,1)\braket{-1,1}{1,0} = a_+(1,-1)\braket{0,1}{1,1} = 1,
\end{equation}
then
\begin{equation}
    \braket{-1,1}{1,0} = \frac{1}{\sqrt{2}}.
\end{equation}
Our result is
\begin{mybox}
\begin{equation}
    \ket{1,0} = \frac{1}{\sqrt{2}}\Bigg[\ket{-1,1} - \ket{1,-1} \Bigg].
\end{equation}
\end{mybox}

Now let's obtain the expansion of \(\ket{1,-1}\). Take \(M=-1\), \(m_1=-1\) and the upper relation:
\begin{equation}
    a_+(1,-1)\braket{0,-1}{1,-1} = -a_+(1,-1)\braket{-1,0}{1,-1}
\end{equation}
Since these are the only non-vanishing CG coefficients, due to normalisation we readily obtain
\begin{equation}
    \braket{-1,0}{1,-1} = -\braket{0,-1}{1,-1} = \frac{1}{\sqrt{2}},
\end{equation}
and then
\begin{mybox}
\begin{equation}
    \ket{1,-1} = \frac{1}{\sqrt{2}}\Bigg[\ket{-1,0} - \ket{0,-1} \Bigg].
\end{equation}
\end{mybox}

When \(J=0\) we have \(M=0\). Let's find the expansion of \(\ket{0,0}\). Consider \(m_1=-1\) and the upper relation
\begin{equation}
    a_+(1,-1)\braket{0,0}{0,0} = -a_+(1,0)\braket{-1,1}{0,0},
\end{equation}
and now \(m_1=1\) with the lower relation
\begin{equation}
    a_-(1,1)\braket{0,0}{0,0} = -a_-(1,0)\braket{1,-1}{0,0}.
\end{equation}
Since these are the three allowed CG coefficients, using the normalisation condition it's straightforward to obtain
\begin{mybox}
\begin{equation}
    \ket{0,0} = \frac{1}{\sqrt{3}}\Bigg[ \ket{-1,1} - \ket{0,0} + \ket{1,-1}\Bigg].
\end{equation}
\end{mybox}

Let's find the expansion of \(\ket{2,M}\). Let's take \(M=2\), \(m_1=1\) and the upper relation:
\begin{equation}
    a_-(2,2)\braket{1,0}{2,1} = a_+(1,0)\braket{1,1}{2,2}.
\end{equation}
When \(J=M=2\) there is only one CG coefficient, \(m_1=m_2=1\), so due to normalisation we know that
\begin{equation}
    \braket{1,1}{2,2} = 1,
\end{equation}
then
\begin{mybox}
\begin{equation}
    \ket{2,2} = \ket{1,1},
\end{equation}
\end{mybox}
and also
\begin{equation}
    \braket{1,0}{2,1} = \frac{1}{\sqrt{2}}.
\end{equation}
Taking \(m_1=0\) and the upper relation:
\begin{equation}
    a_-(2,2)\braket{0,1}{2,1} = a_+(1,0)\braket{1,1}{2,2} =a_+(1,0),
\end{equation}
then
\begin{equation}
    \braket{0,1}{2,1} = \frac{1}{\sqrt{2}}.
\end{equation}
With \(M=1\) there are only two coefficients, one with \((m_1=1,m_2=0)\) and another with \((m_1=0,m_2=1)\). Since we've already found them, we have
\begin{mybox}
\begin{equation}
    \ket{2,1} = \frac{1}{\sqrt{2}}\Bigg[\ket{0,1} + \ket{1,0}  \Bigg].
\end{equation}
\end{mybox}

Taking \(M=0\), \(m_1=1\) and the lower relation we obtain
\begin{equation}\label{eq:bla1}
    a_+(2,0)\braket{1,0}{2,1} = a_-(1,1)\braket{0,0}{2,0} + a_-(1,0)\braket{1,-1}{2,0},
\end{equation}
now with \(m_1=0\) and the lower relation:
\begin{equation}\label{eq:bla2}
    a_+(2,0)\braket{0,1}{2,1} = a_-(1,0)\braket{-1,1}{2,0} + a_-(1,1)\braket{0,0}{2,0},
\end{equation}
then with \(m_1=0\) and the upper relation:
\begin{equation}
    a_-(2,0)\braket{0,-1}{2,-1} = a_+(1,0)\braket{1,-1}{2,0}.
\end{equation}

Taking \(M=-1\) and \(m_1=0\) with the lower relation:
\begin{equation}\label{eq:that}
    a_+(2,-1)\braket{0,0}{2,0} = a_-(1,0)\braket{-1,0}{2,-1} + a_-(1,0)\braket{0,-1}{2,-1},
\end{equation}
and now with \(m_1=-1\) and the upper relation
\begin{equation}\label{eq:this}
    a_-(2,-1)\braket{-1,-1}{2,-2} = a_+(1,-1)\braket{0,-1}{2,-1} + a_+(1,-1)\braket{-1,0}{2,-1}.
\end{equation}

With \(M=-2\) there is only one CG coefficient, with \((m_1=-1,m_2=-1)\), so due to normalisation we know that
\begin{equation}
    \braket{-1,-1}{2,-2} = 1,
\end{equation}
then
\begin{mybox}
\begin{equation}
    \ket{2,-2} = \ket{-1,-1}.
\end{equation}
\end{mybox}

Knowing this, we have from \eqref{eq:this}:
\begin{equation}
    \frac{2}{\sqrt{2}} = \braket{0,-1}{2,-1} + \braket{-1,0}{2,-1}.
\end{equation}
and since with \(M=-1\) there are only two CG coefficients, one with \((m_1=0,m_2=-1)\) and other with \((m_1=-1,m_2=0)\), due to normalisation we know that
\begin{equation}
    \braket{0,-1}{2,-1} = \braket{-1,0}{2,-1} = \frac{1}{\sqrt{2}},
\end{equation}
then
\begin{mybox}
\begin{equation}
    \ket{2,-1} = \frac{1}{\sqrt{2}}\Bigg[ \ket{0,-1} + \ket{-1,0}\Bigg].
\end{equation}
\end{mybox}

With \eqref{eq:that} we obtain
\begin{equation}
    \braket{0,0}{2,0} = \frac{2}{\sqrt{6}},
\end{equation}
and finally with \eqref{eq:bla1} and \eqref{eq:bla2} we obtain
\begin{equation}
    \braket{1,-1}{2,0} = \braket{-1,1}{2,0} = \frac{1}{\sqrt{6}},
\end{equation}
then finally
\begin{mybox}
\begin{equation}
    \ket{2,0} = \frac{1}{\sqrt{6}}\Bigg[\ket{-1,1} + 2 \ket{0,0} + \ket{1,-1} \Bigg].
\end{equation}
\end{mybox}


\chapter*{2.}

We have the radial equation with \(l=0\):
\begin{equation}
    -\frac{1}{2m}\dv[2]{U_{E0}}{r} - \frac{a^2}{8}\exp{-\frac{r}{r_0}}U_{E0} = E U_{E0}
\end{equation}
which we can write as
\begin{equation}
    - \dv[2]{U_{E0}}{r} - \frac{ma^2}{4}\exp{-\frac{r}{r_0}}U_{E0} = 2mE U_{E0}
\end{equation}

Defining
\begin{equation}
    z = ar_0\sqrt{m}\exp{-\frac{r}{2r_0}} \Rightarrow \frac{z^2}{r_0^2} = ma^2\exp{-\frac{r}{2r_0}},
\end{equation}
we have
\begin{equation}
    \dv{r} = -\frac{1}{2r_0} z \dv{z},
\end{equation}
and also
\begin{equation}
    \dv[2]{r} = \frac{1}{4r_0^2}\left(z \dv{z} + z^2 \dv[2]{z} \right).
\end{equation}
Then our radial equation becomes
\begin{equation}
    -\frac{1}{4r_0^2}\left(z \dv{U}{z} + z^2\dv[2]{U}{z} \right) - \frac{z^2}{4r_0^2} U = 2mE U,
\end{equation}
which can be written as
\begin{equation}
    \left( z \dv{U}{z} + z^2\dv[2]{U}{z} + z^2 + (8mEr_0^2) \right)U = 0,
\end{equation}
and defining 
\begin{equation}
    n = i2r_0 \sqrt{2mE},
\end{equation}
we obtain 
\begin{equation}
    \left[ z \dv{U}{z} + z^2\dv[2]{U}{z} + (z^2 -n^2) \right]U = 0,
\end{equation}
which is a \emph{Bessel equation}. The solutions are, then,
\begin{equation}
    U_{E0}(z) = c_1 J_n(z) + c_2 Y_n(z),
\end{equation}
where \(J_n\) is a Bessel function of the first kind and \(Y_n\) is a Bessel function of the second kind.

We have the boundary conditions
\begin{equation}
    U_{E0}(r=0) = U_{E0}(r\to \infty)=0.
\end{equation}
The situation \(r\to 0\) is equivalent to \(z \to ar_0\sqrt{m}\), and the situation \(r\to \infty\) is equivalent to \(z \to 0\), so the boundary conditions on \(U_{E0}(z)\) become
\begin{equation}
    U_{E0}(z\to ar_0\sqrt{m}) = 0 \qq{and} U_{E0}(z\to 0) = 0.
\end{equation}
Since \(Y_n(z\to 0)\) diverges, we must set \(c_2 = 0\). Also, we obtain the condition
\begin{equation}
    J_n(ar_0\sqrt{m}) = 0,
\end{equation}
implying that
\begin{equation}
    ar_0\sqrt{m} = j_{n,k},
\end{equation}
where \(j_{n,k}\) is the k-th root of the Bessel function with index \(n\):
\begin{equation}
    mr_0^2 = \left(\frac{j_{n,k}}{a}\right)^2.
\end{equation}

Then, the radial solution is
\begin{equation}
    R_{E0}(r) = c_1 \frac{J_n(ar_0\sqrt{m}\exp{-\frac{r}{2r_0}})}{r},
\end{equation}
and the full solution is
\begin{equation}
    \psi_{E00}(\vb{r}) = c_1\frac{J_n(ar_0\sqrt{m}\exp{-\frac{r}{2r_0}})}{r} Y_{00}(\theta,\phi) = c_1\frac{J_n(ar_0\sqrt{m}\exp{-\frac{r}{2r_0}})}{\sqrt{4\pi r^2}} ,
\end{equation}
with eigenvalues
\begin{equation}
    E_k = - \frac{n^2}{8mr_0^2} = - \frac{n^2a^2}{8j^2_{n,k}}.
\end{equation}
















% \backmatter
% \printbib
\end{document}