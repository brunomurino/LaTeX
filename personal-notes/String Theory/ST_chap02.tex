\documentclass[oneside, 12pt]{book}

\usepackage{mypreamble}
\usepackage[backend=biber,style=nature]{biblatex}
%\cite{barton} = [1] and \footfullcite[p. 324]{barton} = ^1 + footnote, p. 324 (e.g.)

\usepackage{mycommands}
\usepackage{mytheme1}

\addbibresource{ref_ST.bib}

\begin{document}

\chapter{Relativistic String}

A string has one spatial dimension, therefore it needs two parameters to describe its world-manifold, and, that being the case, this manifold is a surface called the \textit{world-sheet}. Henceforth, the invariant interval is the area element of this world-sheet.\par
The parameters \(\{ \sigma\}\) we use are \(\sigma^0 = \tau\) and \(\sigma^1 = \sigma\), \(\tau\) being a timelike parameter, and \(\sigma\) a spacelike parameter.\par 
To denote points of the world-sheet, we'll use the capitalized \(X^{\mu}(\tau,\sigma)\), which are called the \textit{string coordinates}.\par

\section{Equations of motion}

We know that the area element of a surface is just the square-root of the determinant of the induced metric on that surface, that is
\beq[] \dd{A} = \sqrt{-det(h)}\eeq
where
\beq[] h_{\alpha \beta} \equiv \eta_{\mu \nu}\pdv{X^{\mu}}{\sigma^{\alpha}}\pdv{X^{\nu}}{\sigma^{\beta}}\eeq
Computing we find that
\beq[] \sqrt{-det(h)} = \sqrt{(\dot{X}\cdot X')^2 -(\dot{X})^2(X')^2}\eeq
with
\beq[] \dot{X} = \pdv{X}{\tau}\qq{and} X' = \pdv{X}{\sigma}\eeq
Then, the so called Nambu-Goto string action is
\beq[] S[X] = -\frac{T_0}{c}\int \sqrt{-det(h)} = -\frac{T_0}{c}\int \dd{\tau} \dd{\sigma}\sqrt{(\dot{X}\cdot X')^2 -(\dot{X})^2(X')^2}\eeq
therefore the Lagrangian is
\beq[] \mc{L} = -\frac{T_0}{c}\sqrt{(\dot{X}\cdot X')^2 -(\dot{X})^2(X')^2}\eeq\par
The constants introduce are only to match units. We can see that this action is invariant under the Poincaré group, since there are no free indices on its expression, and this action also shows diffeomorphism invariance (\textit{diff}-invariance), this last symmetry is
\beq[] S[X'(\tau ',\sigma ')] = S[X(\tau, \sigma)]\eeq
where
\beq[] \tau '= \tau '(\tau, \sigma)\qq{and} \sigma ' = \sigma ' (\tau, \sigma)\eeq\par 
Varying the Nambu-Goto action, we obtain the following equation of motion for the relativistic string
\beq[] \pdv{\mc{P}^{\tau}_{\mu}}{\tau} + \pdv{\mc{P}^{\sigma}_{\mu}}{\sigma} = 0\eeq
with
\beq[] \mc{P}^{\tau}_{\mu} \equiv \pdv{\mc{L}}{\dot{X}^{\mu}} = -\frac{T_0}{c} \frac{(\dot{X}\cdot X')X'_{\mu} - (X')^2\dot{X}_{\mu}}{\sqrt{(\dot{X}\cdot X')^2 -(\dot{X})^2(X')^2}}\eeq
and
\beq[] \mc{P}^{\sigma}_{\mu} \equiv \pdv{\mc{L}}{X'^{\mu}} = -\frac{T_0}{c} \frac{(\dot{X}\cdot X')\dot{X}_{\mu} - (\dot{X})^2 X'_{\mu}}{\sqrt{(\dot{X}\cdot X')^2 -(\dot{X})^2(X')^2}}\eeq
This expressions can be summarized as
\beq[]
\begin{bmatrix}
\mc{P}^{\tau}_{\mu} \\ \mc{P}^{\sigma}_{\mu}
\end{bmatrix} = \frac{T_0}{c}\frac{1}{\sqrt{-det(h)}}[cof(h)]^{T}
\begin{bmatrix} \dot{X}_{\mu} \\ X'_{\mu}
\end{bmatrix}\eeq
with
\beq[] h = \begin{bmatrix}
(\dot{X})^2 & \dot{X}\cdot X' \\
\dot{X}\cdot X' & (X')^2 \end{bmatrix}\qq{and} [cof(h)]^T = det(h)h^{-1} = \begin{bmatrix}
(\dot{X})^2 & -\dot{X}\cdot X' \\
-\dot{X}\cdot X' & (X')^2 \end{bmatrix}\eeq\par 

When varying the action, we happen to have boundary terms, which we imposed to be zero. Thus we cannot impose any boundary condition. There are two of them that are natural
\begin{itemize}
    \item The Dirichlet boundary condition, or fixed endpoints boundary condition
    \beq[] \pdv{X^{\mu}}{\tau}=0\qq{at endpoints,} \mu\neq 0\eeq
    \item The free endpoint condition
    \beq[] \mc{P}^{\sigma}_{\mu} = 0 \qq{at endpoints}\eeq
\end{itemize}

\section{World-sheet currents}

For the variation given by
\beq[] \var{X^{\mu}} = \epsilon^{\mu}\eeq
we get that the currents are
\beq[] j^{\alpha} = \left(\mc{P}^{\tau}_{\mu},\mc{P}^{\sigma}_{\mu} \right) \eeq 
and the conservation of currents is just
\beq[] \pdv{\mc{P}^{\tau}_{\mu}}{\tau} + \pdv{\mc{P}^{\sigma}_{\mu}}{\sigma} = 0\eeq
which is just the equation of motion for the string.\par 
We can integrate this currents and get the conserved charge
\beq[]  p_{\mu}(\tau) = \int _{0}^{\sigma_1}\mc{P}^{\tau}_{\mu}\dd{\sigma}\eeq
which is the spacetime momentum carried by the string, since the symmetry observed corresponds to a spacetime translational invariance. We can conclude then that \(\mc{P}^{\tau}_{\mu} \) is the \(\sigma\) density of spacetime momentum carried by the string. And using the equations of motion we have that
\beq[] \dv{p_{\mu}}{\tau} = -\eval{\mc{P}^{\sigma}_{\mu}}^{\sigma_1}_{0} \eeq
where \(\sigma_1\) is one endpoint of the string. Noting that for closed strings \(\sigma_1\) corresponds to the same point \(0\), and that for open strings, if we impose free boundary conditions, we have that \( \mc{P}^{\sigma}_{\mu}=0\), we get that the momentum is conserved with respect to \(\tau\)
\beq[] \dv{p_{\mu}}{\tau}=0\eeq
It's important to note that if we impose Dirichlet boundary conditions, fixed endpoints, spacetime momentum is no conserved. This result will give rise to the \(D\)-branes, which will be studied later.\par 

\section{Lorentz symmetry}

\section{The slope parameter}

\section{Choosing a parametrization}

The equations of motion we obtained before were nasty, but we can make them simpler by choosing an appropriate gauge. We will try to impose the following gauge on \(\tau\)
\beq[] n\cdot X = \frac{2\pi\hbar c^2}{\sigma_1}\alpha' (n\cdot p) \tau \eeq
and the following gauge on \( \sigma\)
\beq[] n\cdot p = \sigma_1 \left(n \cdot \mc{P}^{\tau}\right)\eeq
where \( n_{\mu}\) is an arbitrary vector and \( \sigma \in [0,\sigma_1]\), for both open string and closed string. It's common to set \( \sigma \in [0,2\pi]\) for closed strings and \(\sigma \in [0,\pi]\) for open strings.\par

In the \(\tau\) gauge we want 
\beq[] \dv{\tau} n\cdot p = 0\eeq
and for that to happen we must impose
\beq[] n\cdot \mc{P}^{\sigma} = 0\qq{at both endpoints}\eeq
or for the closed string
\beq[] n\cdot \mc{P}^{\sigma} = 0\qq{at some point} \eeq
Now for the \(\sigma\) gauge we want to impose
\beq[] \dv{\sigma}n\cdot \mc{P}^{\tau}=0 \eeq
this condition implies the \(\sigma\) gauge stated above, since \( \mc{P}^{\tau}\) is the spacetime momentum density along the string.\par 

We can dot the equations of motion with \(n\), leading us to
\beq[] \pdv{\tau}\left(n\cdot \mc{P}^{\tau}\right) + \pdv{\sigma}\left(n\cdot\mc{P}^{\sigma}\right)  = 0\eeq
\beq[] \pdv{\tau}\left(\frac{n\cdot p }{\sigma_1}\right) + \pdv{\sigma}\left(n\cdot\mc{P}^{\sigma}\right)  = 0 \eeq
\beq[] \pdv{\sigma} \left(n\cdot \mc{P}^{\sigma} \right) = 0\eeq
but already we imposed \( n\cdot \mc{P}^{\sigma} = 0\) at endpoints, and now with this last equation we know that this holds at every point of the string, this means
\beq[] n\cdot \mc{P}^{\sigma} = 0 \qq{for all} \sigma \in [0,\sigma_1]\eeq
Now, the consequences of the parametrization are
\beq[] \dot{X}\cdot X' = 0 \qq{and} \dot{X}^2 + X'^2 = 0\eeq
Which can be summarized as the pair of equations
\beq[] \left(\dot{X}\pm X' \right)^2 =0 \eeq
And leads to
\beq[] \mc{P}^{\tau} = \frac{T_0}{c}\dot{X} \qq{and} \mc{P}^{\sigma} = -\frac{T_0}{c}X'\eeq
Therefore the equations of motion become
\beq[] \ddot{X} - X'' = 0\eeq
Which is just the wave equation.\par 

\section{Wave equation}

Solving the wave equation for an open string with free boundary conditions, and with \( \sigma \in [0,\pi]\), we obtain
\beq[] X^{\mu}(\tau,\sigma) = x^{\mu}_0 + 2\alpha'p^{\mu}\tau - i\sqrt{2\alpha'}\sum_{n=1}^{\infty}\left( a^{\mu*}_n e^{in\tau} - a^{\mu}_ne^{-in\tau}\right)\frac{\cos(n\sigma)}{\sqrt{n}}\eeq
And if we perform the following substitutions
\beq[eq:alpha0pmu] \alpha^{\mu}_0 = \sqrt{2\alpha'}p^{\mu}\eeq
\beq[] \alpha^{\mu}_n = a^{\mu}_n\sqrt{n}\qc \alpha^{\mu}_{-n} = a^{\mu *}_{n}\sqrt{n}\qc n\geq 1\eeq
Noting that
\beq[] \alpha^{\mu}_{-n} = \left(a^{\mu}_n\right)^*\eeq
We get
\beq[] X^{\mu}\left( \tau,\sigma \right) = x^{\mu}_0 + \sqrt{2\alpha'}\alpha^{\mu}_0 \tau + i\sqrt{2\alpha'}\sum_{n\neq0}\frac{1}{n}\alpha^{\mu}_n e^{-in\tau}\cos(n\sigma)\eeq\par 
It is convenient to record the derivatives of \(X^{\mu}\)
\beq[] \dot{X}^{\mu} = \sqrt{2\alpha'}\sum_{n\in\mb{Z}}\alpha^{\mu}_n\cos{n\sigma} e^{-in\tau}\eeq
\beq[] {X^{\mu}}' = -i\sqrt{2\alpha'}\sum_{n\in\mb{Z}}\alpha^{\mu}_n\sin{n\sigma} e^{-in\tau}\eeq
And also
\beq[] \dot{X}^{\mu} \pm {X^{\mu}}' =\sqrt{2\alpha'}\sum_{n\in\mb{Z}}\alpha^{\mu}_ne^{-in\left( \tau\pm\sigma\right)}\eeq

\section{Solutions in light-cone coordinates}

If we set \(n_{\mu} = \left( \frac{1}{\sqrt{2}},\frac{1}{\sqrt{2}},0,...,0\right)\), then we have the following parametrization
\beq[] X^+(\tau, \sigma) = \beta \alpha'p^+ \tau \qq{and} p^+ = \frac{2\pi}{\beta}\mc{P}^{\tau +} \eeq
With \(\beta = 2\) for open strings and \(\beta =1 \) for closed strings, noting that the second equation means that the density of \(p^+\) is constant along the string.\par 

In light-cone coordinates it's useful to separate \(X^{\mu}\) in three parts: \(X^+\), \(X^-\), and \(X^I = \left( X^2, X^3,...,X^d\right)\), where \(X^I\) are called the \textit{traverse coordinates}.\par 

Recalling the general consequence of our parametrization
\beq[] \left(\dot{X} \pm X' \right)^2=0\eeq
In light-cone coordinates, after the substitution
\beq[] \dot{X}^+ = \beta \alpha' p^+  \qq{and} {X^+ }' = 0\eeq
We have
\beq[eq:lcconstrainxdotxprime] \dot{X}^- \pm {X^-} ' = \frac{1}{\beta \alpha '}\frac{1}{2p^+}\left(\dot{X}^I\pm {X^I} ' \right)^2\eeq
which, given \(X^I\), gives us
\beq[] \dot{X}^- \qq{and} {X^-}'\eeq
therefore, upon integration, we find \( X^-\). Recalling that the integration produces a constant \(x_0^-\).\par 
Regarding closed strings, as \(\dd{X^-}\) is an exact differential, we can choose to integrate over a contour of constant \(\tau\), and since the extremes of integration represent the same point on space, such integral must equal zero, thus requiring that
\beq[] \int_{0}^{2\pi} \dd{\sigma}{X^-}' =0\eeq
which is another consistency condition for closed strings.\par
Since the light-cone coordinates are just a particular choice of parametrization corresponding to \(n = \left(\frac{1}{\sqrt{2}},\frac{1}{\sqrt{2}},0,...,0 \right)\), we know that each component of the string coordinates must satisfy
\beq[] X^{\mu}\left( \tau,\sigma \right) = x^{\mu}_0 + \sqrt{2\alpha'}\alpha^{\mu}_0 \tau + i\sqrt{2\alpha'}\sum_{n\neq0}\frac{1}{n}\alpha^{\mu}_n e^{-in\tau}\cos(n\sigma)\eeq
therefore we have \(X^I\) once we fix \(x^I_0\) and \( \alpha^I_n\) for all \(n\).\par 
Knowing that \(X^-\) also satisfies the solution above, we can find the relation
\beq[] \sqrt{2\alpha'}\alpha^-_n = \frac{1}{p^+}L^{\perp}_{n}\eeq
where \(L^{\perp}_n\) is called the \textit{traverse Virasoro} mode
\beq[eq:traversevirasoromode_n] L^{\perp}_n = \hlf \sum_{p\in\mb{Z}}\alpha^I_{n-p}\alpha^I_{p}\eeq
Note that, using \eqref{eq:alpha0pmu}, we have
\beq[eq:L0p+p-] L_0^{\perp} = p^{+}\sqrt{2\alpha'}\alpha_0^{-} = 2\alpha'p^{+}p^{-} \eeq

\section{To summarize}
The particular choice of gauge
\beq[] n\cdot X = \beta\alpha' (n\cdot p) \tau \eeq
\beq[] n\cdot p = \frac{2\pi}{\beta} \left(n \cdot \mc{P}^{\tau}\right)\eeq
results in the constrains
\beq[] \left(\dot{X} \pm X' \right)^2=0\eeq
and in the equation of motion
\beq[eq:stringwaveeom] \ddot{X} - X'' = 0\eeq
of which the solutions are, for free endpoints boundary conditions in the case of open strings, and for every case of closed strings,
\beq[eq:stringsol] X^{\mu}\left( \tau,\sigma \right) = x^{\mu}_0 + \sqrt{2\alpha'}\alpha^{\mu}_0 \tau + i\sqrt{2\alpha'}\sum_{n\neq0}\frac{1}{n}\alpha^{\mu}_n e^{-in\tau}\cos(n\sigma)\eeq
with \(\beta =2\) for open strings and \(\beta =1\) for closed ones, recalling that
\beq[eq:alphamomentum] \alpha^{\mu}_0 = \sqrt{2\alpha'}p^{\mu}\eeq
and that \( \sigma = [0,2 \pi/\beta ]\). Therefore, the full solution is specified by the values \( x^{\mu}_0\), \( \alpha^{\mu}_0\), \( \alpha^{\mu}_n\) and \( \alpha^{\mu}_{-n}\) for \( n=1,2,...,\infty\).\par
Light-cone coordinates
\beq[] X^+(\tau, \sigma) = \beta \alpha'p^+ \tau \qq{and} p^+ = \frac{2\pi}{\beta}\mc{P}^{\tau +} \eeq
satisfy such gauge and give
\beq[] x^{+}_0 =0 \qc \alpha^{+}_{n} = \alpha^{+}_{-n} =0 \qq{with} p^{+} \sim \alpha^{+}_0\qq{free}\eeq
and this fully determines \(X^{+}\).\par 
The solution of \( X^I\) is unconstrained, therefore we need all the parameters
\beq[] x^{I}_0\qc p^I\qc \alpha^I_n \qq{and} \alpha^I_{-n}\qc n=1,2,...,\infty\eeq\par 
The solution of \( X^{-}\) is constrained by the choice of gauge, resulting in the following relation between the parameters
\beq[eq:stringLmode] \alpha^{-}_n = \frac{1}{\sqrt{2\alpha'}p^+}L^{\perp}_{n}\qq{for all} n \in \mb{Z}\qq{where} L^{\perp}_{n} = \hlf \sum_{p\in\mb{Z}}\alpha^I_{n-p}\alpha^I_p\eeq
and free \(x^{-}_0\).\par 
Therefore, to fully find a solution \( X^{\mu}\), we must fix the following parameters
\begin{tcolorbox}
    \beq[eq:stringdynvar] p^{+}\qc x_0^{-}\qc  x_0^I\qc \alpha^I_n\eeq
\end{tcolorbox}
or
\begin{tcolorbox}
    \beq[eq:stringdynvar2] p^{+}\qc x_0^{-}\qc  x_0^I\qc \alpha^I_n\eeq
\end{tcolorbox}

Also, we calculate the mass of the string using the relation 
\beq[] p^2 + M^2 =0\eeq
and find
\beq[eq:stringclassicalmass] M^2 = \frac{1}{\alpha'}\sum_{n=1}^{\infty}a^{I*}_n a^I_n\eeq
where \(M\), by convention, is taken to be always positive.\par

\end{document}