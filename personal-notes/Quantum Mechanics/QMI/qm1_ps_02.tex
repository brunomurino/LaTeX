\documentclass{_mypackages/monograph}

\title{Quantum Mechanics I \\ Problem Set 02} % \MyTitle
\author{Bruno Murino - 8944901} % \MyAuthor
\date{\today} % \MyDate

\addbibresource{qinfo.bib}
\graphicspath{ {figures/} }

\begin{document}
% \frontmatter

\solutionstp
% \dominitoc
% \doparttoc
% \pagestyle{onlypagenum}
% \tableofcontents
% \mainmatter

\chapter*{1.}

% Let there be a composite system \(sR\), of systems \(s\) and \(R\), whose density matrix is \(\rho\). I want \(s\) to be in a pure state, what should be the form of \(\rho\)? Well, first I should compute the reduced density matrix of \(s\), i.e. \(\rho_s = \Tr_R(\rho)\), and then compute the respective purity, i.e. \(\mathcal{P}_s = \Tr(\rho_s^2)\). I said I wanted \(s\) to be in a pure state, this means that \(\mathcal{P}_s\) must be equal \(1\), i.e. \(\mathcal{P}_s=1\). However, since \(\rho_s\) \emph{is}, in fact, a density matrix, we know that \(\Tr(\rho_s) =1\). So we can equate the purity of \(s\) to the trace of \(\rho_s\) and find that
% \begin{equation}
%     \Tr(\rho_s^2) = \Tr(\rho_s),
% \end{equation}
% implying that
% \begin{equation}
%     \rho_s^2 = \rho_s,
% \end{equation}
% which is precisely the definition of a projection operator. By definition, then, we can say that
% \begin{equation}
%     \rho = \rho_s \tens \rho_R + \chi,
% \end{equation}
% where \(\chi\) are terms that have null trace, meaning that they don't interfere with \(\rho_s\) or \(\rho_R\), altough they can represent some correlation between them.

Consider an arbitrary \(\rho\) for \(sR\). Since \(\rho^\dagger = \rho\), i.e. \(\rho\) is Hermitian, \(\rho\) admits the spectral decomposition
\begin{equation}
    \rho = \sum_k \rho_k \ketbra{\phi_k}.
\end{equation}
Since \(\ketbra{\phi_k}\) is an element of the product Hilbert space \(\mathcal{H}_s\tens \mathcal{H}_R\), we can expand it a basis such as \(\ket{a_ib_j}\), implying that
\begin{equation}\label{eq:rhoinab}
    \ket{\phi_k} = \sum_{i,j} c^k_{i,j}\ket{a_ib_j},
\end{equation}
and then
\begin{equation}\label{eq:rhoexp}
    \rho = \sum_{k,i,j,i',j'} \rho_k c^k_{i,j}(c^k_{i',j'})^* \ketbra{a_ib_j}{a_{i'}b_{j'}}.
\end{equation}
If we recall that
\begin{equation}
    \Tr_R\big(\ketbra{a_ib_j}{a_{i'}b_{j'}}\big) = \delta_{j,j'}\ketbra{a_i}{a_{i'}},
\end{equation}
we can take the partial trace over \(R\) of \(\rho\) to find \(\rho_s\)
\begin{equation}
    \rho_s = \sum_{k,i,j,i'} \rho_k c^k_{i,j}(c^k_{i',j})^* \ketbra{a_i}{a_{i'}}.
\end{equation}
If we now impose \(\rho_s\) to be a pure state, then it must be 
\begin{equation}
    \rho_s = \ketbra{\psi}
\end{equation}
for some \(\ket{\psi}\). If, back at \eqref{eq:rhoinab}, we choose a basis \(\ket{a_i b_j}\) such that, for instance, \(\ket{a_1}=\ket{\psi}\), then it must be true that the only non-zero \(c\) is \(c^k_{1,j}\). Knowing this, we can go back to \eqref{eq:rhoexp} and find that
\begin{equation}
    \rho = \sum_{k,j,j'} \rho_k c^k_{1,j}(c^k_{1,j'})^* \ketbra{a_1b_j}{a_1b_{j'}},
\end{equation}
which we can then factor as
\begin{equation}
    \rho = \ketbra{a_1}\tens \sum_{k,j,j'} \rho_k c^k_{1,j}(c^k_{1,j'})^* \ketbra{b_j}{b_{j'}}.
\end{equation}
Since \(\ketbra{a_1}^2 = \ketbra{a_1}\ketbra{a_1} = \ket{a_1}\braket{a_1}\bra{a_1} = \ketbra{a_1}\), we can see that \(\rho_s=\ketbra{a_1}\) is a projection operator.













































% \backmatter
% \printbib
\end{document}