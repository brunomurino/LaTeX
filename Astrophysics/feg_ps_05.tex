\documentclass{_mypackages/monograph}

\title{Formation and Evolution of Galaxies \\ Problem Set 05 \\ Environment} % \MyTitle
\author{Bruno Murino - 8944901} % \MyAuthor
\date{\today} % \MyDate

\addbibresource{qinfo.bib}
\graphicspath{ {figures/} }

\begin{document}
% \frontmatter

\solutionstp
% \dominitoc
% \doparttoc
% \pagestyle{onlypagenum}
% \tableofcontents
% \mainmatter

\chapter*{Tidal Stripping}


Let \(d\) be the distance between the Milky Way, of mass \(M\sim 10^{12}\,\,\si{\solarmass}\), and the LMC, of mass \(m \sim 10^{10}\,\,\si{\solarmass}\) and radius \(r\sim 2.15\,\,\si{\kilo\parsec}\). If
\begin{equation}
    d < r \left(\frac{2M}{m}\right)^{1/3}.
\end{equation}
then tidal stripping will occur with the satellite. Plugging the values we find that
\begin{equation}
    d < 12.6 \,\,\si{\kilo\parsec}.
\end{equation}




\chapter*{Dynamical Friction}


Assume that the Milky Way matter distribution can be described by a
singular isothermal sphere (SIS), with a density profile
\begin{equation}
    \rho(r) = \frac{\sigma^2}{2\pi G r^2}
\end{equation}
where \(\sigma\) is the velocity dispersion (constant = isothermal). Consider a satellite of mass \(M\) at a circular orbit initially at radius \(r_0\) to the Galaxy centre.

\section*{(1)}

For a satellite orbiting at radius \(r_0\), only masses with \(r<r_0\) contribute gravitationally. For the satellite, we have the equation
\begin{equation}
    \frac{G m M}{r_0^2} = M\frac{v_c^2}{r_0},
\end{equation}
where
\begin{equation}
    m = \int_0^{r_0} \rho(r) \dd{V} = \frac{2\sigma^2 r_0}{G}
\end{equation}
is the total mass of the galaxy that contributes and \(v_c\) is the circular velocity. Plugging this result, we find that
\begin{equation}
    v_c^2 = \frac{G }{r_0} m = \frac{G }{r_0}\frac{2\sigma^2 r_0}{G} =  2\sigma^2,
\end{equation}
and then
\begin{equation}\label{eq:vcsigma}
    v_c = \sqrt{2} \sigma,
\end{equation}
from which we see that the rotation curves are flat, i.e. they don't depend on \(r_0\).

\section*{(2)}

Let's consider the Chandrasekhar formula for dynamical friction, which can be written as
\begin{equation}\label{eq:df}
    \vb{F}_{df} = -\frac{4\pi \ln{\Lambda} G^2 \rho M^2 F(X)}{v_c^2} \vu{v}_c,
\end{equation}
where
\begin{equation}
    F(X) = \text{erf}(X) - \frac{2 X\exp{-X^2}}{\sqrt{\pi}} \qc X = \frac{v_c}{\sqrt{2}\sigma}.
\end{equation}
If we plug \eqref{eq:vcsigma} on the \eqref{eq:df} we readily obtain \(X=1\),
\begin{equation}
    \frac{\rho}{v_c^2} = \frac{\sigma^2}{2\pi G r^2 v_c^2} = \frac{1}{4\pi G r^2},
\end{equation}
and then
\begin{equation}
    \vb{F}_{df} = -C\ln{\Lambda} \frac{ G  M^2 }{r^2} \vu{v}_c \qc C = F(1) \approx 0.415.
\end{equation}

\section*{(3)}

In order to find the torque \(\tau\) the dynamical friction exerts on the satellite, let's recall the definition of torque:
\begin{equation}
    \tau = \vb{r}\cross \vb{F} = \norm{\vb{r}} \norm{\vb{F}}\sin\theta.
\end{equation}
Since the orbit of the satellite is always circular, \(\vu{r}\cdot \vu{v_c} = 0\), meaning that \(\theta = \pi/2\), then \(\sin \theta = 1\). Denoting \(\norm{\vb{r}} = r\), we find that the torque we seek is
\begin{equation}
    \tau = r\Big( -C\ln{\Lambda} \frac{ G  M^2 }{r^2}\Big) = -C\ln{\Lambda} \frac{ G  M^2 }{r}.
\end{equation}

Since
\begin{equation}
    \tau = \dv{L}{t},
\end{equation}
and
\begin{equation}
    L = M r v_c,
\end{equation}
for circular motion, we find that
\begin{equation}
    \tau = M v_c \dot{r} = -C\ln{\Lambda} \frac{ G  M^2 }{r},
\end{equation}
thus implying that
\begin{equation}
    \dot{r} = -C\ln{\Lambda} \frac{ G  M }{rv_c},
\end{equation}
and then
\begin{equation}
    r\dd{r} = \Big( -C\ln{\Lambda} \frac{ G  M }{v_c}\Big) \dd{t},
\end{equation}
which we can integrate from \((t=0,r=r_0)\) to \((t_{df},r=0)\),
\begin{equation}
    \int_{r_0}^0 r \dd{r} = -C\ln{\Lambda} \frac{ G  M }{v_c} \int_0^{t_{df}} \dd{t},
\end{equation}
and find that
\begin{equation}
    -\frac{r_0^2}{2} = -C\ln{\Lambda} \frac{ G  M }{v_c} t_{df},
\end{equation}
which leads to
\begin{equation}
    t_{df} = \frac{1}{2C\ln{\Lambda}}\frac{r_0^2 v_c}{GM} = \frac{v_c}{2CG \ln{\Lambda}}\,\,\frac{r_0^2}{M} = K \frac{r_0^2}{M},
\end{equation}
which is how long it takes for the satellite to reach the galactic centre.

\section*{(4)}

Plugging
\begin{equation}
    v_c = \si{220}{\kilo\meter\per\second} \qc \ln{\Lambda}=3 \qand C = 0.428
\end{equation}
we find that
\begin{equation}
    K = \frac{v_c}{2CG \ln{\Lambda}} = \frac{220\,  10^{14}}{6.674} \si{\kilogram\second\per\metre\squared} \approx 3.3\cross 10^{15}\si{\kilogram\second\per\metre\squared} 
\end{equation}
which we can write as
\begin{equation}
    K \approx  15.84\cross 10^{23}\si{\solarmass\second\per\kilo\parsec\squared}.
\end{equation}

Computing \(t_{df}\) for some systems:
\begin{itemize}
    \item Large Magellanic Cloud (LMC): \(M = 10^{10} \si{\solarmass}\), \(r_0 = 50 \si{\kilo\parsec}\):
    \begin{equation}
        t_{df} \approx 4\cross 10^{17} \si{\second} \approx 12.55 \si{\giga\year} \approx 0.90\, H_0^{-1}
    \end{equation}
    \item Sculptor Dwarf galaxy: \(M = 10^{9} \si{\solarmass}\), \(r_0 = 80 \si{\kilo\parsec}\):
    \begin{equation}
        t_{df} \approx 1.01\cross 10^{19} \si{\second} \approx 321.42 \si{\giga\year} \approx 22.95\, H_0^{-1}
    \end{equation}
    \item \(\omega\) Cen: \(M = 2.5\cross 10^6 \si{\solarmass}\), \(r_0 = 6.4 \si{\kilo\parsec}\):
    \begin{equation}
        t_{df} \approx 2.6\cross 10^{19} \si{\second} \approx 822.84 \si{\giga\year} \approx 58.78\, H_0^{-1}
    \end{equation}
    \item Globular Clusters in the MW bulge: \(M = 10^{5} \si{\solarmass}\), \(r_0 = 1.5 \si{\kilo\parsec}\):
    \begin{equation}
        t_{df} \approx 3.56\cross 10^{19} \si{\second} \approx 1129.99 \si{\giga\year} \approx 80.71\, H_0^{-1}
    \end{equation}
    \item The Sun: \(M = 1 \si{\solarmass}\), \(r_0 = 8 \si{\kilo\parsec}\):
    \begin{equation}
        t_{df} \approx 1.01\cross 10^{26} \si{\second} \approx 3.21\cross 10^9 \si{\giga\year} \approx 2.30\cross 10^8\, H_0^{-1}
    \end{equation}
\end{itemize}





\chapter*{Ram Pressure}


The condition for ram-pressure stripping is
\begin{equation}
    \rho_{ICM} > \frac{2\pi G \Sigma_* \Sigma_{gas}}{v^2},
\end{equation}
where \(\rho_{ICM}\) is the gas density of the cluster in which the galaxy is plunging with speed \(v\), and \(\Sigma_*\) and \(\Sigma_{gas}\) are the surface density of stars and gas, respectively, of the galaxy.

According to "Mo", the Milky Way has
\begin{equation}
    M_* = 5\cross 10^{10} \si{\solarmass} \qand M_{gas} = 5\cross 10^9 \si{\solarmass},
\end{equation}
spreed on a surface of radius \(10 \si{\kilo\parsec}\), which implies that
\begin{equation}
\begin{split}
    \Sigma_* &= \frac{M_*}{\pi r^2} = 1.6 \cross 10^2 \,\,\si{\solarmass\per\parsec\squared}, \\
    \Sigma_{gas} &= \frac{M_{gas}}{\pi r^2} = 1.6 \cross 10 \,\,\si{\solarmass\per\parsec\squared},
\end{split}
\end{equation}
and thus
\begin{equation}
    \rho_{ICM} > 70\cross 10^{-6}\,\, \si{\solarmass\per\cubic\parsec},
\end{equation}
and since
\begin{equation}
    \si{\solarmass\per\cubic\parsec} \approx 0.07 \cross 10^{-21} \si{\gram\per\cubic\centi\metre},
\end{equation}
we find that
\begin{equation}
    \rho_{ICM} > 5.1 \cross 10^{-27}\,\, \si{\gram\per\cubic\centi\metre}.
\end{equation}


Assuming that the cluster hot gas distribution follows a \(\beta\) model,
\begin{equation}
    \rho_{ICM} = \rho_0 \Big[1+\left( \frac{r}{r_c}\right)^2 \Big]^{-3\beta/2},
\end{equation}
with \(\beta=2/3\) and \(r_c=0.1\,\, \si{\mega\parsec}\). If
\begin{equation}
    \rho_{ICM}(1.5\,\,\si{\mega\parsec}) = 0.004424\,\, \rho_0  = \frac{3\,M_{ICM}}{4 \pi (1.5\,\,\si{\mega\parsec} )^3},
\end{equation}
with \(M_{ICM} = 6\cross 10^{13}\,\,\si{\solarmass}\), then
\begin{equation}
    \rho_0 \approx 16\cross 10^{-5} \si{\solarmass\per\cubic\parsec} \approx 12\cross 10^{-27} \si{\gram\per\cubic\centi\metre}.
\end{equation}

To find at which \(r^*\) we have \(\rho^*_{ICM} = 5.1 \cross 10^{-27}\,\, \si{\gram\per\cubic\centi\metre}\) we simply solve for \(r^*\) the following
\begin{equation}
    \frac{\rho^*_{ICM}}{\rho_0} = \frac{r_c^2}{r_c^2 + (r^*)^2}.
\end{equation}
Denoting
\begin{equation}
    \frac{\rho^*_{ICM}}{\rho_0} = K \approx 0.425,
\end{equation}
we find that
\begin{equation}
    r^* = r_c \sqrt{\frac{1-K}{K}} \approx 0.11\,\, \si{\mega\parsec}
\end{equation}




% \backmatter
% \printbib
\end{document}