\documentclass[oneside, 12pt]{book}

\usepackage{../mypreamble}
% \usepackage[backend=biber,style=nature]{biblatex}
%\cite{barton} = [1] and \footfullcite[p. 324]{barton} = ^1 + footnote, p. 324 (e.g.)

\usepackage{../mycommands}
\usepackage{../mytheme1}

% \addbibresource{ref_ST.bib}

%--------------------------------------------------------------



%--------------------------------------------------------------

\begin{document}

\pagestyle{mypage2} \normalfont


\chapter{Further developments}

\section{T-duality}

\section{String charges}

\section{Thermodynamics}

\section{Superstrings}

\section{AdS/CFT correspondence}

\section{Parallel D\textit{p}- and D\textit{q}-branes }

If \(p \neq q\), then there will be \(p-q\) dimensions in which an endpoint is free and the other one is not. In this case we call the corresponding coordinates \textit{mixed coordinates}, otherwise the coordinates are either \textit{common tangential coordinates} or \textit{common normal coordinates}. We usually do the following:
\beq[] X^0, X^1,...,X^q \qq{collectively denoted as} X^i\eeq
are common tangential coordinates,
\beq[] X^{q+1},X^{q+2},...,X^{p} \qq{collectively denoted as} X^r\eeq
are mixed coordinates and
\beq[] X^{p+1},X^{p+2},...,X^d \qq{collectively denoted as} X^a\eeq
are common normal coordinates, and of course we must solve them separately.\par

\subsection{Mixed coordinates}

Imposing the mixed boundary conditions
\beq[] \eval{\pdv{X^r}{\sigma}(\tau,\sigma)}_{\sigma=0} = 0 \qq{and} \eval{X^r(\tau,\sigma)}_{\sigma=\pi} = \bar{x}^r_2 \eeq
on the starting point \eqref{eq:waveeqgensol} we find that
\beq[] f^r(u+2\pi) = -f^r(u) \eeq
which implies that the expansion of \(f^r\) is
\beq[] f^r(u) = \sum_{n\in\mb{Z}^{+}_{odd}} \left(F^r_n\cos{\frac{nu}{2}} + H^r_n \sin{\frac{nu}{2}}\right) \eeq
The solution to mixed coordinates then, already using the conventional notation, is
\beq[] X^r(\tau,\sigma) = \bar{x}^r_2 + i\sqrt{2\alpha'} \sum_{n\in\mb{Z}_{odd}}\frac{2}{n}\alpha^r_{\frac{n}{2}} \exp{-i\frac{n}{2}\tau}\cos{\frac{n\sigma}{2}} \eeq
with
\beq[] \alpha^r_{-\frac{n}{2}} = \left(\alpha^r_{\frac{n}{2}}\right)^{*} \eeq\par

Plugging our solution into \eqref{eq:pgauge} we find that the mixed coordinates do not carry any average momentum \(p^r\).\par

Lets also record
\beq[] \dot{X}^r \pm X'^r = \sqrt{2\alpha'} \sum_{n\in \mb{Z}_{odd}} \alpha^r_{\frac{n}{2}}\exp{-i\frac{n}{2}(\tau\pm\sigma)} \eeq










\nocite{*}
\end{document}
