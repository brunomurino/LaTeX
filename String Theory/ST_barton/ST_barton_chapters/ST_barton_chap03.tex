\documentclass[12pt]{report}
\input{preamble}
\begin{document}

\chapter{Electromagnetism and gravitation in various dimensions}
\newpage
\section{Classical electrodynamics}
\begin{itemize}
    \item With $\rho$ being the charge density and $\vb{j}$ being the current density, using the Heaviside-Lorentz system of units, the Maxwell's equations are
    \begin{center}
    Source-free Maxwell's equations
    \end{center}
    $$\curl{\vb{E}} = -\frac{1}{c}\pdv{\vb{B}}{t}$$
    $$\div{\vb{B}} = 0 $$
    \begin{center}
    Maxwell's equations with sources
    \end{center}
    $$\div{\vb{E}} = \rho$$
    $$\curl{\vb{B}} = \frac{1}{c}\vb{j} + \frac{1}{c}\pdv{\vb{E}}{t}$$
    \item The Lorentz force law is
    $$\dv{\vb{P}}{t} = q(\vb{E} + \frac{\vb{v}}{c}\cross \vb{B})$$
    where $\vb{P}$ is the relativistic momentum.
    \item We can write the fields in terms of potentials
    $$\vb{B} = \curl{\vb{A}}$$
    $$\vb{E} = -\frac{1}{c}\pdv{\vb{A}}{t} - \grad{\Phi}$$
    \item  The Gauge transformations are transformations of the potentials that leave the fields unchanged
    $$\Phi \longrightarrow \Phi ' = \Phi - \frac{1}{c}\pdv{\epsilon}{t}$$
    $$\vb{A} \longrightarrow \vb{A'} = \vb{A} + \grad{\epsilon}$$
    where $\epsilon$, the gauge parameter, is \textbf{any} function of space and time. We say that $\vb{E}$ and $\vb{B}$ are gauge invariant.
    \item If two sets of potentials $(\Phi, \vb{A})$ and $(\Phi ', \vb{A'})$ are related by gauge transformations, they are \textbf{physically} equivalent. It can happen that two sets of potentials give the same fields, but there's no function $\epsilon$ that serves as a gauge parameter. In such case, the potentials are physically different, not equivalent. This can occur in spacetimes with compact spatial dimensions, however, it does not happen in Minkowski space.
    \item When there are compact spatial dimensions, a set of potentials defined on patches is used, but the regions overlapped must be related by gauge transformations.
\end{itemize}
\section{Electromagnetism in three dimensions}
\begin{itemize}
    \item The presence of a dimension is expressed if there's a function that depends on a coordinate of that dimension. So, in order to produce electromagnetism in only three spacetime coordinates, we must make that no component of any quantity depends on a coordinate of that dimension.
    \item To drop a third dimension, say $z$, no quantity should have $z$-dependence. A brief analysis (of the Lorentz force law) gives us that $E_z = B_x = B_y = 0$ and $E_x$, $E_y$ and $B_z$ can only depend on $x$ and $y$. It's easy to find that these conditions are consistent with Maxwell's equations.
\end{itemize}
\section{Manifestly relativistic electrodynamics}
\begin{itemize}
    \item The potential four-vector is $$A^{\mu } = (\Phi, A^1, A^2, A^3)$$
    \item Using the notation $\partial _{\mu} \equiv \pdv{x^{\mu }}$
    \item We can create the field strength
    $$F_{\mu \nu } \equiv \partial_ \mu A_{\nu } - \partial _ \nu A_ {\mu }$$
    and, as it's easy to check, $F_{\mu \nu}$ is antisymmetric
    $$F_{\mu \nu } = - F_{\nu \mu }$$
    %\item Denoting $\vu{i}$ the direction of the $i$ axis, we can represent the electric field as $$\vb{E} = -F_{0i}\ \vu{i} = F^{0i}\ \vu{i}$$%
    \item $F^{0i} = -F_{0i}$ and $F^{ij} = F_{ij}$
    \item The gauge transformations can be represented as 
    $$A_{\mu} \longrightarrow A_{\mu}' = A_{\mu}+ \partial_{\mu}\epsilon$$
    where the gauge parameter $\epsilon$ is a function of spacetime.
    \item The field strength $F_{\mu \nu}$ is  gauge invariant.
    \item The source-free Maxwell's equations can be writen as $$T_{\lambda \mu \nu} = \partial_{\lambda}F_{\mu \nu} + \partial_{\mu}F_{\nu \lambda} + \partial_{\nu}F_{\lambda \mu} = 0$$
    Note that $T_{\lambda \mu \nu}$ is totally antisymmetric
    \item If an object, with however many indices, changes sign under the transposition of any two indices, it is said to be totally antisymmetric.
    \item With
    $$j^{\mu} = (c\rho, j^1, j^2, j^3)$$
    we define the current four-vector.
    \item The Maxwell's equations with source can be writen as 
    $$\partial_ {\nu }{F^{\mu \nu }} = \frac{1}{c}j^{\mu}$$
    \item The equations 
    $$\partial_{\lambda}F_{\mu \nu} + \partial_{\mu}F_{\nu \lambda} + \partial_{\nu}F_{\lambda \mu} = 0$$
    $$\partial_ {\nu }{F^{\mu \nu }} = \frac{1}{c}j^{\mu}$$
    define the Maxwell theory in arbitrary dimensions. In $d$ spatial dimensions, the potential four-vector has components $(\Phi, \vb{A})$ where $\vb{A}$ is a $d$-dimensional spatial vector (each component is a spacetime function, always). \item In arbitrary $d$ dimensions, we have that $E_i \equiv F^{0i}$ and $\vb{E}$ is a vector with $d$ components.
    \item In arbitrary $d$ dimensions, the magnetic field has $\frac{d^2 - d}{2}$ components, which don't fit into a spatial vector, as the electric field does.
\end{itemize}
\section{An aside on spheres in higher dimensions}
\begin{itemize}
    \item A n-ball is filled, a n-sphere is not. Yet, we calculate their volume. The sphere covers the ball. In a world with $d$ spatial dimensions, the "greatest" sphere is a $(d-1)$-sphere, whereas the "greatest" ball is a $d$-ball. The $(d-1)$-sphere covers the $d$-ball.
    \item The volume of a sphere of radius $R$ is 
    $$vol(S^{d-1}(R)) = R^{d-1}vol(S^{d-1})$$
    where $vol(S^{d-1})$ is the volume of a sphere of unit radius.
    \item To find the formulas for the volumes, we can evaluate the integral
    $$I_d = \int_ {\mathbb{R}^d} \dd{x_{1}} \dd{x_{2}} \cdots \dd{x_{d}} e^{-r^2}$$
    in two different ways.
    \item We get that 
    $$vol(S^{d-1}) = \frac{2\pi^{\frac{d}{2}}}{\Gamma (\frac{d}{2})}$$
    \item And we get, also
    $$vol(B^d) = \frac{\pi^{d/2}}{\Gamma(1 + \frac{d}{2})}$$
    \item Some properties of the gamma functions are
    $$\Gamma (x+1)=x\Gamma(x) \qq{, } x>0 $$
    $$\Gamma (n) = (n-1)!\qq{, } n \in \mathbb{Z}\qq{and}n\geq1$$
    $$\Gamma(1/2) = \sqrt{\pi}$$
\end{itemize}
\section{Electric fields in higher dimensions}
\begin{itemize}
    \item 
\end{itemize}
\section{Gravitation and Planck's length}
\section{Gravitational potentials}
\section{The Planck length in various dimensions}
\section{Gravitational constants and compactification}
\section{Large extra dimensions}
\section{Problems}
\subsection*{P 3.1}

\subsection*{P 3.2}

\subsection{}


\end{document}