\documentclass{___mymonograph}

\title{Topology} % \MyTitle
\author{Bruno Murino} % \MyAuthor
\date{\today} % \MyDate

\addbibresource{___mainbibfile.bib}

%--------------------------------------------------------------------------------------------------

\begin{document}

\chapter{Topology}
\minitoc

\section{Introdution}

The word 'topology' comes from the Greek "\texttau ó\textpi o\textvarsigma", meaning place, and "\textlambda ó\textgamma o\textvarsigma", meaning study. 

The term 'topology' was coined by Johann Listing\footnote{Johann Benedict Listing (25 July 1808 – 24 December 1882), a German mathematician.} in his famous article \fullcite{listing1848}.

\begin{definition}[Topology] Let \(X\) be any set and let \(\mc{T}\in \mathbb{P}(X)\footnote{Here \(\mathbb{P}(X)\) denotes the \emph{power set} of \(X\).}\), i.e. any collection of subsets of \(X\). Then \(\mc{T}\) is said to be a \emph{topology on \(X\)} if
\begin{enumerate}
    \item \(\emptyset,X \in \mc{T}\).
    \item If \(J\) is any set of indexes, then \(\set{U_j \in \mc{T} | j \in J}\) satisfies \(\bigcup_{j\in J} U_j \in \mc{T}\).
    \item If \(K\) is any \emph{finite} set of indexes, then \(\set{U_k \in \mc{T}| k \in K}\) satisfies \(\bigcap_{k\in K} U_k \in \mc{T}\).
\end{enumerate}
Some remarks on this are the following
\begin{enumerate}
    \item The elements of \(\mc{T}\) are called \emph{open sets}.
    \item It is also said that \(\mc{T}\) \emph{gives a topology} to \(X\).
\end{enumerate}
\end{definition}

\begin{definition}[Topological space] A non-empty set \(X\) endowed with a topology \(\mc{T}\) on \(X\) is called a \emph{topological space} and is denoted by \((X,\mc{T})\).
\end{definition}

\begin{definition}[Neighborhood] Let \((X,\mc{T})\) be a topological space and \(p\in X\). Any set \(U\in \mc{T}\) containing \(p\), i.e. \(p\in U\), is called a \emph{neighborhood} of \(p\).
\end{definition}

An important property of topological spaces is that named after \emph{Hausdorff\footnote{Felix Hausdorff (November 8, 1868 – January 26, 1942), a German mathematician.}}.

\begin{definition}[Hausdorff property] Let \((X,\mc{T})\) be a topological space. If for all \(x,y\in X\) there exists two open sets with respect to \(\mc{T}\), \(A_x\) and \(A_y\), such that \(x \in A_x\) and \(y\in A_y\) but \(A_x \cap A_y = \emptyset\), then the topological space \((X,\mc{T})\) is said to have the \emph{Hausdorff property}.

Topological spaces having the Hausdorff property are called \emph{Hausdorff spaces}, \emph{separated spaces} or even \emph{\(T_2\) spaces}, the latter being related to the \emph{separation axioms}.

\end{definition}


% \backmatter
% \printbib
\end{document}
