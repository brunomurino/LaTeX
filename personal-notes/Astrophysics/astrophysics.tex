\documentclass{_mypackages/monograph}

\title{Astrophysics} % \MyTitle
\author{Bruno Murino} % \MyAuthor
\date{\today} % \MyDate

\addbibresource{cosmology.bib}
\graphicspath{ {figures/} }

\begin{document}
\frontmatter

\monographtp
\dominitoc
\doparttoc
\pagestyle{onlypagenum}
\tableofcontents
\mainmatter

% \chapter{Introduction}

\subsubsection{The Schechter function}

The function defined by
\begin{equation}
    \phi(x)\dd{x} = \phi^* x^\alpha \exp{-x}\dd{x}
\end{equation}
is called the \emph{Schechter function}.


\subsubsection{Magnitude}

The light emitted from the surface of a celestial object must travel across some portion of the universe so it can reach us, and then, be measured by us. However, since there's not only vacuum between a celestial object and us, some fraction of the original light is lost, thus the only thing we can measure is this deprecated light, i.e. deprecated luminosity flux. But before we go on, lets introduce the concept of \emph{magnitude}, first proposed by Hipparchus\footnote{Hipparkhos Nikealainen (c.190 – c.120 bc), greek.} or Ptolemy\footnote{Claudius Ptolemaeus (AD 100 – c.170), greco-roman.} about 2000 years ago, but which modern definition we owe to N. Pogson\footnote{Norman Robert Pogson (23 March 1829 – 23 June 1891), english astronomer}. 

Pogson's idea was to create a scale such that, to each factor of \(100\) on the luminosity measured, the new unit called \emph{magnitude} should decrease by \(5\), i.e. a scale following \emph{Pogson's ratio}: \(\sqrt[5]{100} = 2.512\).

\begin{definition}[Magnitude]
Lets suppose we do a measurement of luminosity with the filter \(x\). Let \(S_{x,0}\) be the reference flux for the filter \(x\) and \(S_x\) the flux we measured. Then \(m_x\) defined by
\begin{equation}
    m_x = - 5 \log_{100}\left(\frac{S_x}{S_{x,0}} \right) = - 2.5 \log_{10}\left(\frac{S_x}{S_{x,0}} \right)
\end{equation}
is called the \emph{magnitude} of the celestial object.
\end{definition}

Since this measurement depends on the distance we are from the celestial object, its not an intrinsic property of it.

\begin{definition}[Apparent magnitude]
If the measurement is performed from the Earth, then the magnitude measured is called the \emph{apparent magnitude}.
\end{definition}

\begin{definition}[Absolute magnitude]
If the measurement could be taken exactly \SI{10}{\parsec} away from the celestial body, the magnitude measured would be the \emph{absolute magnitude}. In other words, the absolute magnitude is defined as what we would measure if the measurement could take place exactly \SI{10}{\parsec} away from the celestial body.
\end{definition}

Its not absurd to think that luminosities should obey the inverse square law, i.e. the apparent magnitude should increase by a factor of 4 if the distance increases by a factor of 2. First, lets define the distance modulus.

\begin{definition}[Distance modulus]
The difference \(\mu = m - M\), where \(m\) is the apparent magnitude and \(M\) is the absolute magnitude, is called the \emph{distance modulus}.
\end{definition}

\begin{mybox}
Bearing in mind the inverse square law, we can write the following relation between the distance modulus and the luminosity distance:
\begin{equation}\label{eq:magdis}
    \mu = m - M = 5 \log_{10} \left( \frac{d}{\SI{10}{\parsec}}\right)
\end{equation}
where \(\mu\) is the distance modulus, \(m\) is the apparent magnitude, \(M\) is the absolute magnitude and \(d\) is the luminosity distance.
\end{mybox}






\subsubsection{The Faber-Jackson Relation} 
Faber\footnote{Sandra Moore Faber (December 28, 1944 - ).} and Jackson\footnote{Robert Earl Jackson (1949 - ).}, in 1976\footfullcite{1976ApJ...204..668F}, noticed that the luminosity of an elliptical galaxy is correlated with its central velocity dispersion:
\begin{equation}
    L \propto \sigma_0^\beta \qq{with} beta \sim 4
\end{equation}

Thus if the apparent luminosity of two galaxies with the same central velocity dispersion is given, we can find the ratio between their distance.

\begin{figure}
    \centering
    \includegraphics[width=\textwidth]{Astrophysics/figures/faber-jackson.png}
    \caption{The log of the velocity dispersion vs. V26 for the Coma and Virgo ellipticals. The lines represent median fits, as described in the text, and are separated by 3.75 mag, indicating a distance ratio D(Coma)/D(Virgo) = 5.62. This implies a peculiar Virgocentric velocity of 311 km s-1. Some of the more deviant points are identified. (from \cite{Dressler1984}).}
    \label{fig:faber-jackson}
\end{figure}

From figure \ref{fig:faber-jackson} we can compare the apparent magnitude of the Virgo and Coma clusters when their central velocity dispersion is the same and find it to be around \(\num{3.75}\).

Now, taking equation \eqref{eq:magdis} for both clusters we find
\begin{gather}
    m_V - M_V = 5 \log_{10} \left( \frac{d_V}{\SI{10}{\parsec}}\right) \\
    m_C - M_C = 5 \log_{10} \left( \frac{d_C}{\SI{10}{\parsec}}\right)
\end{gather}
Since we will read the apparent magnitudes from \ref{fig:faber-jackson} when their central velocity distribution is the same, their absolute magnitude will be the same, thus subtracting one equation from the other we find
\begin{equation}
    m_V - m_C - (M_V - M_C) = m_V - m_C = 5 \log_{10} \left( \frac{d_V}{d_C}\right)
\end{equation}
but since \(m_V - m_C = -3.75\), we find that \(\nicefrac{d_C}{d_V} = 5.62\), which means that the Virgo cluster is closer to us than the Coma cluster.

% \chapter{Cosmology}
% \minitoc

\section{Redshift}

\subsection{The age of the universe at given redshift}

In the Einstein-de Sitter model of the universe:

\begin{equation}
    t(z) = \tau_H \int_{z}^{\infty} \frac{\dd{z}'}{(1+z')E(z')}
\end{equation}
where \(E(z) = (1+z)^{3/2}\), thus
\begin{equation}
    t(z) = \tau_H \int_{z}^{\infty} \frac{\dd{z}'}{(1+z')^{5/2}} =  \tau_H \frac{2 }{3(z+1)^{3/2}}
\end{equation}
with \(\tau_H =  H_0^{-1} = \SI{9.78 }{h^{-1} \giga\year}\).

For \(z=20\) and the accepted value of \(h=0.7\), then, the age of the universe is
\begin{equation}
    t(20) = \SI{9.78}{h^{-1} \giga\year}\, \frac{2}{3(21)^{3/2}} = \frac{6.52}{0.7*96.2341} = \SI{0.097}{\giga\year} = \SI{97}{\mega\year}
\end{equation}


\subsubsection{Roche radius}

Consider a satellite orbiting a \emph{primary body}. The primary exerts a tidal force on the satellite, i.e. the patch of the satellite closest to the primary feels a bigger gravitational pull than the rest of the satellite. 

For computation purposes, let's denote this closest patch by a mass \(\mu\), and let's denote the rest of the satellite by a mass \(m\), which we'll consider to be centred on the satellite. Then, regarding only the action of the primary on the satellite, the relative force experienced by \(\mu\) at different distances the so called \emph{tidal force}. We also have to consider the mutual attraction between \(\mu\) and \(m\), however, assuming \(\mu \ll m\), the force \(m\) experiences due to \(\mu\) is negligible.

The tidal force can be written as
\begin{equation}
    F_T = \frac{GMu}{(d-r)^2} - \frac{GMu}{d^2} \approx \frac{2GMu r}{d^3},
\end{equation}
while the usual gravitational force of the bigger part of the satellite over the small part can be written as
\begin{equation}
    F_G = \frac{Gmu}{r^2}.
\end{equation}

We want to find at what distance from the primary the satellite the tidal force will be bigger than the \(F_G\):
\begin{equation}
\begin{split}
    F_T &> F_G \\
    \frac{2GMu r}{d^3} &> \frac{Gmu}{r^2} \\
    2\frac{M}{m} r^3 &> d^3,
\end{split}
\end{equation}
which leads to
\begin{equation}
    d < r \left(\frac{2M}{m}\right)^{1/3},
\end{equation}
or
\begin{equation}
    d < R\left(\frac{2\rho_M}{\rho_m}\right)^{1/3}.
\end{equation}



\backmatter
\printbib
\end{document}
