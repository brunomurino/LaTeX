\documentclass{___mymonograph}

\title{Groups} % \MyTitle
\author{Bruno Murino} % \MyAuthor
\date{\today} % \MyDate

\addbibresource{___mainbibfile.bib}

%--------------------------------------------------------------------------------------------------

\begin{document}

\chapter{Groups}
\minitoc

\section{Axioms}

\begin{definition}[Group]

A set \(G\) endowed with a closed binary operation \(G\cross G \to G\), called a \emph{product} and denoted by juxtaposition, is called a \emph{group} if the following axioms are satisfied:
\begin{enumerate}
    \item \emph{Associativity}. For every \(a,b,c\in G\) holds \((a b) c = a (b c)\).
    \item \emph{Identity element}. There exists \(e\footnote{The letter "e" is used...}\in G\) such that for every \(g\in G\) holds \(e g = g  e = g\).
    \item \emph{Inverse element}. For every \(g\in G\) there exists \(h \in G\) such that \(g h = h  g = e\).
\end{enumerate}
Some elementary remarks are:
\begin{enumerate}
    \item If \(e,e'\in G\) are both identity elements of \(G\), then the following holds
    \begin{subequations}
    \begin{align}
        e e' = e'  e = e', \\
        e' e = e  e' = e,
    \end{align}
    \end{subequations}
    implying that every group has only one identity element.
    \item If \(h_1,h_2\in G\) are both inverse elements of \(g\), then the following holds
    \begin{equation}
        h_1 = h_1 e = h_1 (g h_2) = (h_1 g) h_2 = e  h_2 = h_2,
    \end{equation}
    which implies that every element \(g\) of a group has only one inverse, which we can safely denote \(g^{-1}\).
    \item Since \(e e = e\), follows that \(e^{-1}=e\).
    \item For every \(g\) holds \((g^{-1})^{-1} = g\), since
    \begin{equation}
        (g^{-1})^{-1} = (g^{-1})^{-1}  \Big(g^{-1} g\Big) = \Big((g^{-1})^{-1}  g^{-1}\Big) g = g.
    \end{equation}
    \item Every groups is, trivially, a \emph{quasigroup}, a \emph{loop}, a \emph{semigroup} and a \emph{monoid}.
\end{enumerate}
\end{definition}

\section{Group morphisms}

\begin{definition}[Group homomorphism]
Let \(G\) and \(H\) be groups. A function \(\phi:G\to H\) is called a \emph{group homomorphism} if
\begin{equation}
    \phi(ab)=\phi(a)\phi(b),
\end{equation}
for every \(a,b\in G\).
\end{definition}

Some crucial consequences of the definition of a group homomorphisms are the following:

\begin{proposition}
Let \(G\) and \(H\) be groups with identity elements \(e_G\) and \(e_H\), respectively, and \(\phi\) be a group homomorphism. Then
\begin{equation}
    \phi(e_G) = e_H.
\end{equation}
\end{proposition}
\begin{proof}
For any \(a\in G\) we know that \(a = ae_G\), thus \(\phi(a) = \phi(a)\phi(e_G)\). Since \(\phi(a)\in H\) has an inverse, \(\phi(a)^{-1}\in H\), we can apply it to the left and find that \(e_H = \phi(e_G)\).
\end{proof}

\begin{proposition}
Let \(G\) and \(H\) be groups with identity elements \(e_G\) and \(e_H\), respectively, and \(\phi\) be a group homomorphism. Then
\begin{equation}
    \phi(a^{-1}) = \phi(a)^{-1}.
\end{equation}
\end{proposition}
\begin{proof}
Since \(a^{-1}a = a a^{-1} = e_G\), we know that
\begin{equation}
    \phi(a^{-1})\phi(a) = \phi(a)\phi(a^{-1}) = \phi(e_G) = e_H,
\end{equation}
meaning that \(\phi(a^{-1})\) has the same properties of the \emph{unique} inverse of \(\phi(a)\), \(\phi(a)^{-1}\), implying the equality \(\phi(a^{-1}) = \phi(a)^{-1}\).
\end{proof}

\begin{definition}[Group isomorphism]
Let \(G\) and \(H\) be groups. If the group homomorphism \(\phi:G\to H\) is \emph{bijective}, then it's called a \emph{group isomorphism}. We say, in this case, that \(G\) and \(H\) are isomorphic by \(\phi\) and denote this by \(G\simeq_{\phi}H\)
\end{definition}




\begin{proposition}
Let \(G\) and \(H\) be groups. If \(\phi: G\to H\) is an isomorphism, then so is \(\phi^{-1}:H\to G\).
\end{proposition}
\begin{proof}
Let \(x,y\in H\). Since \(\phi\) is bijective, we know that there are unique \(a,b\in G\) such that \(\phi(a)=x\) and \(\phi(b)=y\). Then
\begin{equation}
    \phi^{-1}(xy) = \phi^{-1}(\phi(a)\phi(b)) = \phi^{-1}(\phi(ab)) = ab = \phi^{-1}(x)\phi^{-1}(y),
\end{equation}
meaning that \(\phi^{-1}:H\to G\) is also an homomorphism, and since it is also bijective, is is an isomorphism.
\end{proof}

\section{Linear groups}

\begin{definition}[General linear group]
Let \(\mat(n,F)\) be the set of all \(n\cross n\) matrices over the field \(F\). The set of \(n\cross n\) matrices over \(F\) that are invertible., i.e.
\begin{equation}
    \GL(n,F) = \set{A\in \mat(n,F)\qc \det{A}\neq 0},
\end{equation}
forms a group called the \emph{general linear group} and is denoted by \(\GL(n,F)\).
\end{definition}

\begin{definition}[Special linear group]
Let \(\mat(n,F)\) be the set of all \(n\cross n\) matrices over the field \(F\). The set of \(n\cross n\) matrices over \(F\) that have determinant equal \(1\)., i.e.
\begin{equation}
    \SL(n,F) = \set{A\in \mat(n,F)\qc \det{A}= 1},
\end{equation}
forms a group called the \emph{special linear group} and is denoted by \(\SL(n,F)\).
\end{definition}

\section{Representation of a group}

\begin{definition}[Representation]
A representation \((\pi,V)\) of a group \(G\) on a vector space \(V\) is a homomorphism
\begin{equation}
    \pi:G\to \GL(n,V),
\end{equation}
where \(n\) is the dimension of \(V\).
\end{definition}

\begin{definition}[Invariant subspace \emph{and} subrepresentation]
Let \((\pi,V)\) be a representation of a group \(G\) and let \(W\subset V\). If \(\pi(g)w \in W\) for every \(g\in G\) and \(w\in W\), then \(W\) is called an \emph{invariant subspace}. The restriction of \(\pi\) to an invariant subspace \(W\) is called a \emph{subrepresentation} and is denoted \((\pi_{|W},W)\).
\end{definition}

\begin{definition}[Trivial invariant subspace \emph{and} trivial subrepresentation]
Any representation of a group has at least two invariant subspaces: when \(W=\set{0}\) and when \(W=V\). These are called \emph{trivial} invariant subspaces and the respective subrepresentations are called \emph{trivial subrepresentations}.
\end{definition}

\begin{definition}[Irreducible representation]
Let \((\pi,V)\) be a representation of a group \(G\). The representation \(\pi\) is called \emph{irreducible} if it has only trivial subrepresentation.
\end{definition}

\begin{definition}[Reducible representation]
Let \((\pi,V)\) be a representation of a group \(G\). If the representation \(\pi\) is not irreducible then it's called \emph{reducible}.
\end{definition}

\begin{equation}
    \inner{2}{3}
\end{equation}


% \backmatter
% \printbib
\end{document}
