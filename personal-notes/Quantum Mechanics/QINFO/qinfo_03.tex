\documentclass{../_mypackages/monograph}

\title{Quantum Information} % \MyTitle
\author{Bruno Murino} % \MyAuthor
\date{\today} % \MyDate

\addbibresource{qinfo.bib}
\graphicspath{ {figures/} }

\begin{document}
% \frontmatter

% \monographtp
% \dominitoc
% \doparttoc
% \pagestyle{onlypagenum}
% \tableofcontents
% \mainmatter

\chapter{Composite systems}
\minitoc

\section{The tensor product}
Until now, all we have considered was a single system. However, we can choose to treat our system as a composition of sub-systems. Notice that this sub-systems can be independent or not.

Let say that our big system \(AB\) is composed of two sub-systems \(A\) and \(B\). Such system \(AB\) is called a \emph{bipartite system}. We can, then, represent a state of \(AB\) as a \emph{tensor product} or \emph{Kronecker product}. 
\begin{mybox}
If a state of \(A\) is \(\ket{i}_A\) and a state of \(B\) is \(\ket{j}_B\), then we say that the corresponding state of \(AB\) is
\begin{equation}
    \ket{i,j}_{AB} = \ket{i}_A \tens \ket{j}_B.
\end{equation}
Analogously, we define operators of \(AB\) as the tensor product between operator of \(A\) and \(B\), such that
\begin{equation}
    (A\tens B)(C\tens D) = (AB\tens CD)
\end{equation}
whenever \(AB\) and \(CD\) both make sense.
\end{mybox}

For instance, the following is a valid operation
\begin{equation}
    (A\tens B)(\ket{\psi}\tens\ket{\phi}) = (A\ket{\psi})\tens(B\ket{\phi}),
\end{equation}
while
\begin{equation}
    (\bra{0}\tens\ket{0})(\sigma_x\tens \sigma_x)
\end{equation}
is not, since while \(\bra{0}\sigma_x\) makes sense, \(\ket{0}\sigma_x\) does not.

To actually compute some tensor product \(A\tens B\) when \(A\) and \(B\) are matrices we do
\begin{equation}
    A\tens B = \mqty(a_{11} B & \cdots & a_{1N} B \\ \vdots & \ddots & \vdots \\a_{M1} B & \cdots & a_{MN} B ),
\end{equation}
so we also have the following:
\begin{equation}
    \ket{0,0} = \mqty(1 \\ 0\\0\\0)\qc \ket{0,1} = \mqty(0 \\ 1\\0\\0)\qc \ket{1,0} = \mqty(0 \\ 0\\1\\0)\qc \ket{1,1} = \mqty(0 \\ 0\\0\\1).
\end{equation}

\section{Entanglement}

If we consider the bipartite system \(AB\) with \(\mathscr{H}_{AB} = \mathscr{H}_A\tens\mathscr{H}_B\), with \(\dim(\mathscr{H}_{AB}) = 4\), a general state \(\ket{\psi}_{AB}\) of \(AB\) can be written as
\begin{equation}
    \ket{\psi}_{AB} = \sum_{i,j} C_{ij}\ket{i,j}_{AB}\qc i,j\in\set{0,1},
\end{equation}
where \(C_{i,j}\) are a set of coefficients and \(\ket{i,j}_{AB}\) are the four vectors of some basis of \(\mathscr{H}_{AB}\).

\begin{mybox}
If we can find states \(\ket{\psi}_A\in\mathscr{H}_A\) and \(\ket{\psi}_B\in\mathscr{H}_B\) such that
\begin{equation}
    \ket{\psi}_{AB} = \ket{\psi}_A\tens \ket{\psi}_B,
\end{equation}
we say that the state \(\ket{\psi}_{AB}\) is a \emph{product state}. If our state is \emph{not} a product state, then we say it is an \emph{entangled state}. Some very important entangled states are the \emph{Bell states}:
\begin{equation}
\begin{split}
    \ket{\Phi_1} &= \frac{1}{\sqrt{2}}\bigg(\ket{0,0} + \ket{1,1} \bigg),\\
    \ket{\Phi_2} &= \frac{1}{\sqrt{2}}\bigg(\ket{0,0} - \ket{1,1} \bigg),\\
    \ket{\Phi_3} &= \frac{1}{\sqrt{2}}\bigg(\ket{0,1} + \ket{1,0} \bigg),\\
    \ket{\Phi_4} &= \frac{1}{\sqrt{2}}\bigg(\ket{0,1} - \ket{1,0} \bigg).
\end{split}
\end{equation}
\end{mybox}



\section{The partial trace}

Consider the bipartite system \(AB\) with \(\ket{a}\) being a basis of system \(A\) and \(\ket{b}\) a basis of system \(B\). Consider now the arbitrary operator on \(AB\)
\begin{equation}
    \mathcal{O} = \sum_\alpha A_\alpha \tens B_\alpha.
\end{equation}
Computing the trace of \(\mathcal{O}\) we find that
\begin{equation}
\begin{split}
    \Tr(\mathcal{O}) &= \sum_\alpha \Tr(A_\alpha \tens B_\alpha).
\end{split}
\end{equation}
Computing only \(\Tr(A_\alpha \tens B_\alpha)\) in the \(\ket{a,b}\) basis, we find that
\begin{equation}
\begin{split}
    \Tr(A_\alpha \tens B_\alpha) &= \sum_{a,b} \ev{A_\alpha \tens B_\alpha}{a,b} \\
    &= \sum_{a,b} (\bra{a}\tens\bra{b})(A_\alpha\tens B_\alpha)(\ket{a}\tens\ket{B}) \\
    &= \sum_a \ev{A_\alpha}{a} \sum_b \ev{B_\alpha}{b} \\
    &= \Tr(A_\alpha)\Tr(B_\alpha),
\end{split}
\end{equation}
and then finally
\begin{equation}
    \Tr(\mathcal{O}) = \sum_\alpha \Tr(A_\alpha)\Tr(B_\alpha).
\end{equation}

We can now define the \emph{partial trace over \(A\)} as
\begin{equation}
    \Tr_A(\mathcal{O}) = \sum_\alpha \Tr(A_\alpha)B_\alpha,
\end{equation}
and analogously for the partial trace over \(B\). 
\begin{mybox}
Moreover, since the trace is a linear operation, we can conclude that
\begin{equation}
    \Tr_A(\ketbra{a,b}{a',b'}) = \delta_{a,a'}\ketbra{b}{b'},
\end{equation}
and analogous for \(\Tr_B\). And also
\begin{equation}
    \Tr(A\tens B) = \Tr_B\Bigg( \Tr_A(A\tens B)\Bigg),
\end{equation}
meaning that we can compute the full trace by taking consecutive partial traces.
\end{mybox}

\section{Reduced density matrix}

By means of the partial trace operation we can define the \emph{reduced density matrices} \(\rho_A\) and \(\rho_B\) of a composite system \(AB\), with \(\rho_{AB}\), as
\begin{equation}
    \rho_A = \Tr_B (\rho_{AB}) \qand \rho_B = \Tr_A (\rho_{AB}),
\end{equation}
and notice that, given an operator \(O\) acting only on \(A\), i.e. \(A = A\tens \idm_A\), we find
\begin{equation}
    \ev{A} = \Tr(A\rho_A)
\end{equation}

\section{The ambiguity of mixtures}

If we have a bipartite qubit system on the Bell state \(\ket{\Phi_1}\), then
\begin{equation}
    \rho_{AB} = \ketbra{\Phi_1} = \frac{1}{2}\bigg(\ketbra{0,0} + \ketbra{0,0}{1,1} + \ketbra{1,1}{0,0} + \ketbra{1,1} \bigg),
\end{equation}
and the important result is that the reduced density matrices \(\rho_A\) and \(\rho_B\) become
\begin{equation}
    \rho_A = \rho_B = \frac{1}{2}\mqty(1 & 0 \\ 0 & 1)
\end{equation}
and then the purity of \(A\), i.e. \(\mathcal{P}_A = \Tr(\rho_A^2)\), is
\begin{equation}
    \mathcal{P}_A = \frac{1}{2} \leq 1,
\end{equation}
meaning that \(\rho_A\) corresponds to a mixed state! This leads to an interesting conclusion:
\begin{mybox}
\emph{If a state of \(AB\) is entangled,
the reduced states of \(A\) and \(B\) are mixed states.} This means that if you measure system \(A\) and find a mixed state, then you can't know if either your system is entangled with another system, or if your system is just a mixture of states.
\end{mybox}

\section{Loss of information by taking the partial trace}

The operation of taking the partial trace is \emph{irreversible} and in general looses information, as will be shown in the example that follows.

Let
\begin{equation}
    \rho_{AB} = \rho_A^0 \tens \rho_B^0 + \chi, \qq{with} \chi = \mqty(0 & 0 & 0 & 0 \\ 0 & 0 & \alpha & 0 \\ 0 & \alpha & 0 & 0 \\ 0 & 0 & 0 & 0).
\end{equation}
Since \(\Tr_A(\chi) = \Tr_B(\chi) = 0\), we find that
\begin{equation}
    \rho_A = \rho_A^0 \qand \rho_B = \rho_B^0,
\end{equation}
just like if \(\chi\) didn't exist!













% \backmatter
% \printbib
\end{document}